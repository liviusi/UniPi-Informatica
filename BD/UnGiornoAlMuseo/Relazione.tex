\documentclass[10pt, italian, openany, landscape, fleqn]{book}
% Set page margins
\usepackage[top=100pt,bottom=100pt,left=68pt,right=66pt]{geometry}

\usepackage[]{graphicx}
% If multiple images are to be added, a folder (path) with all the images can be added here 
\graphicspath{ {images/} }

\usepackage{hyperref}
\hypersetup{
    colorlinks=true,
    linkcolor=blue,
    filecolor=magenta,
    urlcolor=blue,
}
 
% All page numbers positioned at the bottom of the page
\usepackage{fancyhdr}
\usepackage{lscape}
\usepackage{adjustbox}

\newcommand*{\MyIncludeGraphics}[2][]{%
\begin{adjustbox}{max size={0.8\textwidth}{\textheight}}
    \includegraphics[#1]{#2}%
\end{adjustbox}
}
\usepackage{tikz}

\fancyhf{} % clear all header and footers
\fancyfoot[C]{\thepage}
\renewcommand{\headrulewidth}{0pt} % remove the header rule
\pagestyle{fancy}

% Changes the style of chapter headings
\usepackage{titlesec}

\titleformat{\chapter}
   {\normalfont\LARGE\bfseries}{\thechapter.}{1em}{}

% Change distance between chapter header and text
\titlespacing{\chapter}{0pt}{50pt}{2\baselineskip}
\usepackage{titlesec}
\titleformat{\section}
  [hang] % <shape>
  {\normalfont\bfseries\Large} % <format>
  {} % <label>
  {0pt} % <sep>
  {} % <before code>
\renewcommand{\thesection}{} % Remove section references...
\renewcommand{\thesubsection}{\arabic{subsection}} %... from subsections

% Numbered subsections
\setcounter{secnumdepth}{3}

% Prevents LaTeX from filling out a page to the bottom
\raggedbottom

\usepackage{listings}
\usepackage{color}
\usepackage{xcolor}
\usepackage{amsfonts}
\usepackage{amsmath}
\usepackage{forest}
\useforestlibrary{linguistics}
\forestapplylibrarydefaults{linguistics}

% Code Listings
\definecolor{vgreen}{RGB}{104,180,104}
\definecolor{vblue}{RGB}{49,49,255}
\definecolor{vorange}{RGB}{255,143,102}
\definecolor{vlightgrey}{RGB}{245,245,245}
\lstdefinestyle{sql} {
	language=SQL,
	basicstyle=\ttfamily,
	keywordstyle=\color{vblue},
	identifierstyle=\color{black},
	commentstyle=\color{vgreen},
	tabsize=4,
	backgroundcolor = \color{vlightgrey},
	literate=*{:}{:}1
}

\definecolor{codegreen}{rgb}{0,0.6,0}
\definecolor{codegray}{rgb}{0.5,0.5,0.5}
\definecolor{codepurple}{rgb}{0.58,0,0.82}
\definecolor{backcolour}{rgb}{0.95,0.95,0.92}

\begin{document}
\begin{titlepage}
	\clearpage\thispagestyle{empty}
	\centering
	\vspace{1cm}

    \includegraphics[scale=0.60]{unipi-logo.png}
    
	{\normalsize \noindent Dipartimento di Informatica \\
	             Corso di Laurea in Informatica \par}
	
	\vspace{2cm}
	{\Huge \textbf{Un giorno al Museo} \par}
	\vspace{1cm}
	{\large Basi di Dati}

    {\large {05/06/2021 - anno accademico 2020/2021}}

    \begin{minipage}[t]{0.47\textwidth}
    	{\large{ Prof. Giorgio Ghelli}}
    \end{minipage}\hfill\begin{minipage}[t]{0.47\textwidth}\raggedleft
        {\large {Salvatore Correnti \\ 584136 - Corso A\\ }}
        {\large {Domenico Iiripino \\ 599248 - Corso A\\ }}
    	{\large {Giacomo Trapani \\ 600124 - Corso A\\ }}
    \end{minipage}

    \vspace{4cm}

	\pagebreak

\end{titlepage}

% A * after the section/chapter command indicates an unnumbered header which will not be added to the table of contents

\section{Esercizio 1}
Scopo di questo testo \`e descrivere la specifica di \textbf{Un giorno al museo}, un software di gestione del sistema museale
della regione Toscana.

Ogni \textbf{museo} \`e composto da diversi \textbf{ambienti} collegati a coppie tramite \textbf{varchi}, distinti in \textbf{sale} e \textbf{ambienti di servizio}.

Ogni \textbf{sala} pu\`o essere o una \textbf{sala museale} o una \textbf{sala per mostre temporanee} o una \textbf{sala esclusiva} per l'esposizione di opere
esclusive.

Per ogni museo \`e possibile acquistare un \textbf{abbonamento generale}, che permette di visitare in totale un certo numero di sale
museali e per mostre temporanee, oppure un \textbf{abbonamento speciale} che consente di visitare anche le sale esclusive.


Di ogni \textbf{museo} interessano il \textit{nome}, gli \textit{ambienti che lo compongono}, \textit{come sono collegati} fra di loro e gli \textit{utenti che hanno
effettuato almeno una visita} in quel museo.

Di ogni \textbf{ambiente} interessano l'\textit{identificatore}, il \textit{nome} e i \textit{varchi} con cui \`e collegato ad altri ambienti.

Di ogni \textbf{sala} interessa inoltre il \textit{tipo} (sala museale, sala per mostre o sala esclusiva) e gli \textit{utenti} che l'hanno visitata.

Di ogni \textbf{ambiente di servizio} interessa inoltre il \textit{tipo} (bar, biglietteria, ristorante, bagno).

Di ogni \textbf{varco} interessano l'\textit{identificatore}, la \textit{coppia di ambienti} che collega e il \textit{tipo} di varco (scala, porta, apertura).


Di ogni \textbf{utente} interessano \textit{nome}, \textit{cognome}, \textit{indirizzo fisico}, \textit{indirizzo email}, \textit{visite effettuate}, \textit{riviste ricevute} e \textit{abbonamenti acquistati}.

Di ogni \textbf{biglietto} interessano il \textit{codice}, il \textit{tipo} (tariffa bianca o tariffa verde), e le \textit{sale visitate} con quel biglietto.

Di ogni \textbf{visita} (di sala) interessano la \textit{sala visitata}, l'\textit{utente visitatore}, se la visita al museo di cui fa parte
è stata \textit{pagata con} un abbonamento o con un biglietto e l'\textit{ora di inizio e di fine}.

Di ogni \textbf{abbonamento} interessano il \textit{numero di sale che permette di visitare}, il \textit{tempo in minuti} che permette per le visite,
gli \textit{utenti che lo hanno acquistato}, le \textit{visite effettuate} con quell'abbonamento, se l'abbonamento \`e \textit{ancora valido} e se \`e 
\textit{speciale}.

Di ogni \textbf{rivista} interessano il \textit{nome} ("Il museo scientifico" o "Il museo d'arte"), gli \textit{utenti iscritti} e le \textit{spedizioni effettuate}.

Di ogni \textbf{spedizione} interessano la \textit{rivista spedita} e l'\textit{utente} che l'ha ricevuta.

% \MyIncludeGraphics{esercizio1.png}

\pagebreak
\section{Esercizio 2}
% \MyIncludeGraphics{esercizio2.png}
\pagebreak
\subsection*{Vincoli non catturati dallo schema}
\subsubsection*{Vincoli interrelazionali}
\begin{gather*}
(\forall v \in Visite. \ (v.Abbonamento \neq NULL \oplus v.Biglietto \neq NULL)); \\
(\forall v \in Visite \ con \ v.Abbonamento \neq NULL. \ (v.Abbonamento \in v.Utente.Abbonamenti) \wedge (v.Abbonamento.valido = 1)); \\
(\forall v1, v2 \in Visite \ con \ v1.Biglietto, \ v2.Biglietto \neq NULL. \ (v1.Biglietto = v2.Biglietto) \Rightarrow (v1.Sala.Museo = v2.Sala.Museo)); \\
(\forall v \in Visite. \ (v.Abbonamento \neq NULL) \Rightarrow ((v.Sala.tipo = 'esclusiva') \Leftrightarrow (v.Abbonamento.Speciale = 1))); \\
(\forall s \in Spedizioni. \ s.Rivista \in s.Utente.Riviste).
\end{gather*}
\subsubsection*{Vincoli intrarelazionali}
\begin{gather*}
(\forall v \in Varchi. \ v.Ambiente1 \neq v.Ambiente2); \\
(\forall a \in Abbonamenti.\ (a.minuti \geq 0) \wedge (a.numeroSale \geq 0)); \\
(\forall a1 \in Ambienti. \ \neg(\exists a2 \in Ambienti. \ (a1 \neq a2) \wedge (a1.codiceAmbiente = a2.codiceAmbiente)); \\
(\forall v1 \in Varchi. \ \neg(\exists v2 \in Varchi. \ (v1 \neq v2) \wedge (v1.codiceVarco = v2.codiceVarco)); \\
(\forall a1 \in Abbonamenti. \ \neg(\exists a2 \in Abbonamenti. \ (a1 \neq a2) \wedge (a1.codiceAbbonamento = a2.codiceAbbonamento)); \\
(\forall u1 \in Utenti. \ \neg(\exists u2 \in Utenti. \ (u1 \neq u2) \wedge (u1.email = u2.email)); \\
(\forall c \ campo \ dello \ schema. \ c \neq NULL ).
\end{gather*}
\pagebreak

\section{Esercizio 3}
Si pone che oraInizio, oraFine sono dei timestamp, per cui identificano un momento preciso nel tempo.
% \MyIncludeGraphics{esercizio3.png}
\pagebreak
\subsection*{Dipendenze funzionali}
Il simbolo (*) a destra di una dipendenza funzionale indica tutti gli attributi della relazione corrispondente.

e.g. data R(IdR,A,B) vale che (IdR \( \rightarrow  \ *) \Leftrightarrow \) (IdR \( \rightarrow \) IdR, A, B).

Si ricorda che non vale IdAmbiente1, IdAmbiente2 \( \rightarrow \) *, in quanto nel testo non \`e specificato che una coppia di ambienti sia 
sempre collegata da uno e un solo varco.

Si ha che la relazione Visite \textit{non} \`e in forma normale di Boyce-Codd in quanto IdBiglietto,IdAbbonamento\( \rightarrow \)IdUtente. Tutte le altre relazioni sono invece in \textbf{BCNF}
in quanto tutte le dipendenze funzionali individuate sono scomponibili nella forma (X\( \rightarrow \)A) con A attributo e X superchiave.

Si riportano di seguito le dipendenze funzionali; da queste risulta possibile ricavare tutte le altre dipendenze non banali.

\subsubsection*{Musei}
\begin{gather*}
IdMuseo \rightarrow *.
\end{gather*}
\subsubsection*{Ambienti}
\begin{gather*}
codiceAmbiente \rightarrow *; \\
IdAmbiente \rightarrow *.
\end{gather*}
\subsubsection*{Varchi}
\begin{gather*}
codiceVarco \rightarrow *; \\
IdVarco \rightarrow *.
\end{gather*}
\subsubsection*{AmbientiDiServizio}
\begin{gather*}
IdAmbiente \rightarrow *.
\end{gather*}
\subsubsection*{Sale}
\begin{gather*}
IdAmbiente \rightarrow *.
\end{gather*}
\subsubsection*{Visite}
\begin{gather*}
IdBiglietto, \ IdAbbonamento, \ oraInizio, \ oraFine \rightarrow *; \\
IdBiglietto, \ IdAbbonamento, \ oraInizio \rightarrow *; \\
IdBiglietto, \ IdAbbonamento, \ oraFine \rightarrow *; \\
IdBiglietto, \ IdAbbonamento \rightarrow IdUtente; \\
IdUtente, \ oraInizio, \ oraFine \rightarrow *; \\
IdUtente, \ oraInizio \rightarrow *; \\
IdUtente, \ oraFine \rightarrow *; \\
IdVisita \rightarrow *.
\end{gather*}
\subsubsection*{Biglietti}
\begin{gather*}
codiceBiglietto \rightarrow *; \\
IdBiglietto \rightarrow *.
\end{gather*}
\subsubsection*{Abbonamenti}
\begin{gather*}
codiceAbbonamento \rightarrow *; \\
IdAbbonamento \rightarrow *.
\end{gather*}
\subsubsection*{Sottoscrizioni}
\begin{gather*}
IdAbbonamento, \ IdUtente \rightarrow * \\
\end{gather*}
\subsubsection*{Utenti}.
\begin{gather*}
IdUtente \rightarrow *; \\
email \rightarrow *.
\end{gather*}
\subsubsection*{Iscrizioni}
\begin{gather*}
IdUtente, \ IdRivista \rightarrow *.
\end{gather*}
\subsubsection*{Spedizioni}
\begin{gather*}
IdUtente, \ IdRivista \rightarrow *.
\end{gather*}
\subsubsection*{Riviste}
\begin{gather*}
IdRivista \rightarrow *.
\end{gather*}
\pagebreak

\section{Esercizio 4}
\subsection*{Esercizio 4.a}
La query restituisce Id, nome e cognome di tutti e soli gli utenti a cui \`e stata spedita una rivista dopo il 2 Apr. 2021.
\begin{lstlisting}[style=sql]
SELECT DISTINCT
    u.IdUtente, u.nome, u.cognome
FROM
    Utenti u
    JOIN Spedizioni s USING (IdUtente)
WHERE s.data > '2021-04-02'
\end{lstlisting}
\subsection*{Esercizio 4.b}
La query restituisce Id, Nome, Cognome e numero di sale museali visitate di tutti e soli gli utenti
che hanno visitato almeno 3 sale museali ordinandoli sul numero di sale museali visitate (in ordine crescente).
\begin{lstlisting}[style=sql]
SELECT
    u.IdUtente, MAX(u.nome) as NomeUtente, 
    MAX(u.cognome) as CognomeUtente,
    COUNT(*) AS nVisite
FROM
    Utenti u
    JOIN Visite v USING (IdUtente)
    JOIN Sale s USING (IdAmbiente)
WHERE s.tipo = 'museali'
GROUP BY u.IdUtente
HAVING COUNT(*) >= 3
ORDER BY nVisite
\end{lstlisting}
\pagebreak
\subsection*{Esercizio 4.c}
La query restituisce Id, Nome, Cognome e numero di abbonamenti sottoscritti di tutti e soli
gli utenti che hanno sottoscritto almeno 2 abbonamenti speciali.
\begin{lstlisting}[style=sql]
SELECT
    u.IdUtente, MAX(u.nome) AS NomeUtente, MAX(u.cognome)
    AS CognomeUtente, COUNT(*) as nSottoscrizioni
FROM
    Utenti u
    JOIN Sottoscrizioni s USING (IdUtente)
    JOIN Abbonamenti a USING (IdAbbonamento)
WHERE (a.speciale = 1)
GROUP BY u.IdUtente
HAVING COUNT(*) >= 2
\end{lstlisting}
\subsection*{Esercizio 4.d}
La query restitituisce tutti e soli gli utenti che hanno effettuato almeno una visita
in una sala per mostre.
\begin{lstlisting}[style=sql]
SELECT
    u.IdUtente, u.nome, u.cognome
FROM Utenti u
WHERE
    EXISTS
    (
        SELECT *
        FROM
            Visite v
            JOIN Sale s USING (IdAmbiente)
        WHERE
            (v.IdUtente = u.IdUtente) AND (s.tipo = 'per mostre')
    )
\end{lstlisting}
\pagebreak
\subsection*{Esercizio 4.e}
La query restituisce tutti e soli i musei per cui tutte le sale sono museali o per mostre.
\begin{lstlisting}[style=sql]    
SELECT
    m.IdMuseo, m.nome
FROM Musei m
WHERE
    NOT EXISTS
    (
	    SELECT *
	    FROM
            Sale s
	        JOIN (Ambienti a) USING (IdAmbiente)
	    WHERE
            (a.IdMuseo = m.IdMuseo) AND (s.tipo = 'esclusiva')
    )
\end{lstlisting}
\subsection*{Esercizio 4.f}
La query restituisce ID e nome degli di tutti e soli gli utenti che hanno visitato almeno 100 sale.
\begin{lstlisting}[style=sql]    
SELECT
    u.IdUtente, u.nome
FROM Utenti u
WHERE
    100 <=
    (
        SELECT COUNT(*)
	    FROM Visite v
	    WHERE (v.IdUtente = u.IdUtente)
    )
\end{lstlisting}
\pagebreak

\section{Esercizio 5}
\subsection*{Esercizio 5.1.a}
\begin{forest}
[\(\delta\)
    [\( \pi^b_{u.IdUtente, u.Nome, u.Cognome} \)
        [
            [\( \Join_{u.IdUtente = s.IdUtente} \) 
                [\textit{Utenti u}]
            [
                \( \sigma_{s.data \ > \ '2021-04-02'} \)
                    [ \textit{Spedizioni s} ]
            ]
            ]
        ]
    ]
]
\end{forest}
\pagebreak
\subsection*{Esercizio 5.1.b}
\subsubsection*{Versione 1}
\begin{forest}
[
    \( \tau_{nVisite} \)
    [
        \( \pi^b_{u.IdUtente, MAX(u.nome) \ AS \ NomeUtente, MAX(u.cognome) \ AS \ CognomeUtente, COUNT(*) \ AS \ nVisite} \)
        [
            \( \sigma_{COUNT(*) \geq 3} \)
            [
                \( \{ u.IdUtente \} \ \gamma \ \{ MAX(u.nome); MAX(u.cognome); COUNT(*) \} \)
                [
                    \( \Join_{v.IdAmbiente = s.IdAmbiente} \)
                    [
                        \( \Join_{u.IdUtente = v.IdUtente} \)
                        [
                            \textit{Utenti u}
                        ]
                        [
                            \textit{Visite v}
                        ]
                    ]
                    [
                        \( \sigma_{s.tipo = 'museali'} \)
                        [
                            \textit{Sale s}
                        ]
                    ]
                ]
            ]
        ]
    ]
]
\end{forest}

Il calcolo degli attributi (\textit{NomeUtente}, \textit{CognomeUtente}) come (MAX(u.nome), MAX(u.cognome))
- che risulta possibile in quanto il raggruppamento avviene sulla chiave primaria \textit{IdUtente} - 
permette di avere un piano di accesso fisico leggermente pi\`u efficiente.
\pagebreak
\subsubsection*{Versione 2}
\begin{forest}
[
    \( \tau_{nVisite} \)
    [
        \( \pi^b_{u.IdUtente, MAX(u.nome) \ AS \ NomeUtente, MAX(u.cognome) \ AS \ CognomeUtente, COUNT(*) \ AS \ nVisite} \)
        [
            \( \sigma_{COUNT(*) \geq 3} \)
            [
                \( \{ u.IdUtente \} \ \gamma \ \{ MAX(u.nome); MAX(u.cognome); COUNT(*) \} \)
                [
                    \( \Join_{u.IdUtente = v.IdUtente} \)
                    [
                        \textit{Utenti u}
                    ]
                    [
                        \( \Join_{v.IdAmbiente = s.IdAmbiente} \)
                        [
                            \textit{Visite v}
                        ]
                        [
                            \( \sigma_{s.tipo = 'museali'} \)
                            [
                                \textit{Sale s}
                            ]
                        ]
                    ]
                ]
            ]
        ]
    ]
]
\end{forest}

Questo piano logico pu\`o essere tradotto in un piano fisico in cui la Sort prima della GroupBy pu\`o essere eliminata.
\pagebreak
\subsection*{Esercizio 5.1.c}
\subsubsection*{Versione 1}
\begin{forest}
[
    \( \pi^b_{u.IdUtente, MAX(u.nome) \ AS \ NomeUtente, MAX(u.cognome) \ AS \ CognomeUtente, COUNT(*) \ AS \ nSottoscrizioni} \)
    [
        \( \sigma_{COUNT(*) \geq 2} \)
        [
            \( \{ u.IdUtente \} \ \gamma \ \{ MAX(u.nome); MAX(u.cognome); COUNT(*) \} \)
            [
                \( \Join_{u.IdUtente = s.IdUtente} \)
                [
                    \textit{Utenti u}
                ]
                [
                    \( \Join_{s.IdAbbonamento = a.IdAbbonamento} \)
                    [
                        \textit{Sottoscrizioni s}
                    ]
                    [
                        \( \sigma_{a.speciale = 1} \)
                        [
                            \textit{Abbonamenti a}
                        ]
                    ]
                ]
            ]
        ]
    ]
]
\end{forest}

Anche in questo caso risulta possibile scrivere un piano di accesso fisico in cui la Sort prima della GroupBy viene eliminata.
\pagebreak
\subsubsection*{Versione 2}
\begin{forest}
[
    \( \pi^b_{u.IdUtente, MAX(u.nome) \ AS \ NomeUtente, MAX(u.cognome) \ AS \ CognomeUtente, COUNT(*) \ AS \ nSottoscrizioni} \)
    [
        \( \sigma_{COUNT(*) \geq 2} \)
        [
            \( \{ u.IdUtente \} \ \gamma \ \{ MAX(u.nome); MAX(u.cognome); COUNT(*) \} \)
            [
                \( \Join_{a.IdAbbonamento = s.IdAbbonamento} \)
                [
                    \( \Join_{u.IdUtente = s.IdUtente} \)
                    [
                        \textit{Utenti u}
                    ]
                    [
                        \textit{Sottoscrizioni s}
                    ]
                ]
                [
                    \( \sigma_{a.speciale = 1} \)
                    [
                        \textit{Abbonamenti a}
                    ]
                ]
            ]
        ]
    ]
]
\end{forest}
\pagebreak
\subsection*{Esercizio 5.2.a}
Previa ridenominazione per semplificare la scrittura: Utenti \( \rightarrow \) u, Spedizioni \( \rightarrow \) s.

\begin{forest}
[
    Distinct
    [
        Project(u.IdUtente; u.nome; u.cognome)
        [
            SortMerge({$u.IdUtente = s.IdUtente$})
            [
                Sort(\{u.IdUtente\})
                [
                    TableScan(u)
                ]
            ]
            [
                Sort(\{s.IdUtente\})
                [
                    Filter(s.data \( > \) '2021-04-02')
                    [
                        TableScan(s)
                    ]
                ]
            ]
        ]
    ]
]
\end{forest}
\pagebreak
\subsection*{Esercizio 5.2.b}
Previa ridenominazione per semplificare la scrittura: Visite \( \rightarrow \) v, Utenti \( \rightarrow \) u, Sale \( \rightarrow \) s.
\subsubsection*{Versione 1}
\begin{forest}
[
    Sort(nVisite)
    [
        Project(u.IdUtente; MAX(u.nome) AS NomeUtente; MAX(u.cognome) AS CognomeUtente; COUNT(*) AS nVisite)
        [
            Filter(COUNT(*) \( \geq \) 3)
            [
                GroupBy(\{ u.IdUtente \}; \{MAX(u.nome); MAX(u.cognome); COUNT(*)\})
                [
                    Sort(\{u.IdUtente\})
                    [
                        SortMerge({$v.IdAmbiente = s.IdAmbiente$})
                        [
                            Sort(\{v.IdAmbiente\})
                            [
                                SortMerge({$u.IdUtente = v.IdUtente$})
                                [
                                    Sort(\{u.IdUtente\})
                                    [
                                        TableScan(u)
                                    ]
                                ]
                                [
                                    Sort(\{v.IdUtente\})
                                    [
                                        TableScan(v)
                                    ]
                                ]
                            ]
                        ]
                        [
                            Sort(\{s.IdAmbiente\})
                            [
                                Filter({$s.tipo = 'museali'$})
                                [
                                    TableScan(s)
                                ]
                            ]
                        ]
                    ]
                ]
            ]
        ]
    ]
]
\end{forest}
\pagebreak
\subsubsection*{Versione 2}
\begin{forest}
[
    Sort(nVisite)
    [
        Project(u.IdUtente; MAX(u.nome) AS NomeUtente; MAX(u.cognome) AS CognomeUtente; COUNT(*) AS nVisite)
        [
            Filter(COUNT(*) \( \geq \) 3)
            [
                GroupBy(\{ u.IdUtente \}; \{MAX(u.nome); MAX(u.cognome); COUNT(*)\})
                [
                    SortMerge({$v.IdUtente = u.IdUtente$})
                    [
                        Sort(\{u.IdUtente\})
                        [
                            TableScan(u)
                        ]
                    ]
                    [
                        Sort(\{v.IdUtente\})
                        [
                            SortMerge({$v.IdAmbiente = s.IdAmbiente$})
                            [
                                Sort(\{v.IdUtente\})
                                [
                                    TableScan(v)
                                ]
                            ]
                            [
                                Sort(\{s.IdAmbiente\})
                                [
                                    Filter({$s.tipo = 'museali'$})
                                    [
                                        TableScan(s)
                                    ]
                                ]
                            ]
                        ]
                    ]
                ]
            ]
        ]
    ]
]
\end{forest}

Questa versione contiene 14 operatori al posto dei 15 dell'altra; si \`e potuto eliminare la Sort prima della
GroupBy in quanto l'output di SortMerge al quarto livello dell'albero \`e gi\`a ordinato su u.IdUtente per definizione
(dell'operatore).
\pagebreak
\subsection*{Esercizio 5.2.c}
Previa ridenominazione per semplificare la scrittura: Abbonamenti \( \rightarrow \) a, Utenti \( \rightarrow \) u, Sottoscrizioni \( \rightarrow \) s.
\subsubsection*{Versione 1}
\begin{forest}
[
    Project(u.IdUtente; MAX(u.nome) AS NomeUtente; MAX(u.cognome) AS CognomeUtente; COUNT(*) AS nSottoscrizioni)
    [
        Filter(COUNT(*) \( \geq \) 2)
        [
            GroupBy(\{u.IdUtente\}; \{MAX(u.nome); MAX(u.cognome); COUNT(*)\})
            [
                Sort(\{u.IdUtente\})
                [
                    SortMerge({$ s.IdAbbonamento = a.IdAbbonamento $})
                    [
                        Sort(\{ s.IdAbbonamento \})
                        [
                            SortMerge({$ u.IdUtente = s.IdUtente $})
                            [
                                Sort(\{ u.IdUtente \})
                                [
                                    TableScan(u)
                                ]
                            ]
                            [
                                Sort(\{ s.IdUtente \})
                                [
                                    TableScan(s)
                                ]
                            ]
                        ]
                    ]
                    [
                        Sort(\{ a.IdAbbonamento \})
                        [
                            Filter({$ a.speciale = 1 $})
                            [
                                TableScan(a)
                            ]
                        ]
                    ]
                ]
            ]
        ]
    ]
]
\end{forest}
\pagebreak
\subsubsection*{Versione 2}
\begin{forest}
[
    Project(u.IdUtente; MAX(u.nome) AS NomeUtente; MAX(u.cognome) AS CognomeUtente; COUNT(*) AS nSottoscrizioni)
    [
        Filter(COUNT(*) \( \geq \) 2)
        [
            GroupBy(\{u.IdUtente\}; \{MAX(u.nome); MAX(u.cognome); COUNT(*)\})
            [
                SortMerge({$u.IdUtente = s.IdUtente$})
                [
                    Sort(\{u.IdUtente\})
                    [
                        TableScan(u)
                    ]
                ]
                [
                    Sort(\{s.IdUtente\})
                    [
                        SortMerge({$ s.IdAbbonamento = a.IdAbbonamento $})
                        [
                            Sort(\{s.IdAbbonamento\})
                            [
                                TableScan(s)
                            ]
                        ]
                        [
                            Sort(\{a.IdAbbonamento\})
                            [
                                Filter({$a.speciale = 1$})
                                [
                                    TableScan(a)
                                ]
                            ]
                        ]
                    ]
                ]
            ]
        ]
    ]
]
\end{forest}

Questa versione contiene 13 operatori al posto dei 14 dell'altra; anche in questo caso la Sort prima della
GroupBy pu\`o essere eliminata poich\'e l'output di SortMerge al quarto livello dell'albero risulta ordinato su u.IdUtente.
\pagebreak
\subsection*{Esercizio 5.3.a}
Previa ridenominazione per semplificare la scrittura: Utenti \( \rightarrow \) u, Spedizioni \( \rightarrow \) s;
si assumono inoltre gli indici IdxUtentePK su u.IdUtente, IdxUtenteFK su s.IdUtente.

\begin{forest}
[
    Distinct
    [
        Project(u.IdUtente; u.nome; u.cognome)
        [
            IndexNestedLoop({$u.IdUtente = s.IdUtente$})
            [
                TableScan(u)
            ]
            [
                IndexFilter(s; IdxUtenteFK; s.data \( > \) '2021-04-02')
            ]
        ]
    ]
]
\end{forest}

Non si fa uso di IdxUtentePK in quanto un operatore IndexFilter(u, ...) avrebbe condizione \textit{true}
e sarebbe "equivalente" a IndexScan.
\pagebreak
\subsection*{Esercizio 5.3.b}
\subsubsection*{Versione 1}

Previa ridenominazione per semplificare la scrittura: Visite \( \rightarrow \) v, Utenti \( \rightarrow \) u, Sale \( \rightarrow \) s;
si assumono inoltre gli indici IdxvAmbiente su v.IdAmbiente, IdxsAmbiente su s.IdAmbiente.

\begin{forest}
[
    Sort(nVisite)
    [
        Project(u.IdUtente; MAX(u.nome) AS NomeUtente; MAX(u.cognome) AS CognomeUtente; COUNT(*) AS nVisite)
        [
            Filter(COUNT(*) \( \geq \) 3)
            [
                GroupBy(\{ u.IdUtente \}; \{MAX(u.nome); MAX(u.cognome); COUNT(*)\})
                [
                    Sort(\{u.IdUtente\})
                    [
                        IndexNestedLoop({$v.IdAmbiente = s.IdAmbiente$})
                        [
                            [
                                Sort(\{v.IdAmbiente\})
                                [
                                    SortMerge({$u.IdUtente = v.IdUtente$})
                                    [
                                        Sort(\{u.IdUtente\})
                                        [
                                            TableScan(u)
                                        ]
                                    ]
                                    [
                                        Sort(\{v.IdUtente\})
                                        [
                                            TableScan(v)
                                        ]
                                    ]
                                ]
                            ]
                            [
                                IndexFilter(s; IdxsAmbiente; {$s.tipo = 'museali'$})
                            ]
                        ]
                    ]
                ]
            ]
        ]
    ]
]
\end{forest}

Per lo stesso motivo che al punto precedente, non risulta necessario fare uso di IdxvAmbiente.
\pagebreak
\subsubsection*{Versione 2}
Previa ridenominazione per semplificare la scrittura: Visite \( \rightarrow \) v, Utenti \( \rightarrow \) u, Sale \( \rightarrow \) s;
si assumono inoltre gli indici IdxvAmbiente su v.IdAmbiente, IdxsAmbiente su s.IdAmbiente, IdxuUtente su u.IdUtente.

\begin{forest}
[
    Sort(nVisite)
    [
        Project(u.IdUtente; MAX(u.nome) AS NomeUtente; MAX(u.cognome) AS CognomeUtente; COUNT(*) AS nVisite)
        [
            Filter(COUNT(*) \( \geq \) 3)
            [
                GroupBy(\{ u.IdUtente \}; \{MAX(u.nome); MAX(u.cognome); COUNT(*)\})
                [
                    SortMerge({$v.IdUtente = u.IdUtente$})
                    [
                        Sort(\{u.IdUtente\})
                        [
                            TableScan(u)
                        ]
                    ]
                    [
                        Sort(\{v.IdUtente\})
                        [
                            IndexNestedLoop({$v.IdAmbiente = s.IdAmbiente$})
                            [
                                TableScan(v)
                            ]
                            [
                                IndexFilter(s; IdxsAmbiente; {$s.tipo = 'museali'$})
                            ]
                        ]
                    ]
                ]
            ]
        ]
    ]
]
\end{forest}

La Sort prima della GroupBy pu\`o essere eliminata e - come prima - non si fa uso di IdxvAmbiente per lo stesso motivo.
Non si ricorre neanche a IdxuUtente in quanto \textit{non} \`e sulla relazione interna; anche invertendo le due met\`a del sottoalbero
si ripresenterebbe il problema di non avere condizioni su u: non risulta dunque possibile convertire SortMerge in una IndexNestedLoop.
\pagebreak
\subsection*{Esercizio 5.3.c}
Previa ridenominazione per semplificare la scrittura: Abbonamenti \( \rightarrow \) a, 
Utenti \( \rightarrow \) u, Sottoscrizioni \( \rightarrow \) s; si assumono inoltre gli indici IdxuUtente su u.IdUtente, IdxsAbb su s.IdAbbonamento e 
IdxaAbb su a.IdAbbonamento.
\subsubsection*{Versione 1}
\begin{forest}
[
    Project(u.IdUtente; MAX(u.nome) AS NomeUtente; MAX(u.cognome) AS CognomeUtente; COUNT(*) AS nSottoscrizioni)
    [
        Filter(COUNT(*) \( \geq \) 2)
        [
            GroupBy(\{u.IdUtente\}; \{MAX(u.nome); MAX(u.cognome); COUNT(*)\})
            [
                SortMerge({$u.IdUtente = s.IdUtente$})
                [
                    Sort(\{u.IdUtente\})
                    [
                        TableScan(u)
                    ]
                ]
                [
                    Sort(\{ s.IdUtente \})
                    [
                        IndexNestedLoop({$ s.IdAbbonamento = a.IdAbbonamento $})
                        [
                            TableScan(s)
                        ]
                        [
                            IndexFilter(a; IdxaAbb; {$a.speciale = 1$})
                        ]
                    ]
                ]
            ]
        ]
    ]
]
\end{forest}

La Sort prima della GroupBy viene eliminata, si fa uso di un solo indice per gli stessi motivi che al punto (b); SortMerge non risulta convertibile in IndexNestedLoop.
\pagebreak
\subsubsection*{Versione 2}
\begin{forest}
[
    Project(u.IdUtente; MAX(u.nome) AS NomeUtente; MAX(u.cognome) AS CognomeUtente; COUNT(*) AS nSottoscrizioni)
    [
        Filter(COUNT(*) \( \geq \) 2)
        [
            GroupBy(\{u.IdUtente\}; \{MAX(u.nome); MAX(u.cognome); COUNT(*)\})
            [
                Sort(\{u.IdUtente\})
                [
                    IndexNestedLoop({$ s.IdAbbonamento = a.IdAbbonamento $})
                    [
                        SortMerge({$ s.IdUtente = u.IdUtente $})
                        [
                            Sort(\{ u.IdUtente \})
                            [
                                TableScan(u)
                            ]
                        ]
                        [
                            Sort(\{ s.IdUtente \})
                            [
                                TableScan(s)
                            ]
                        ]
                    ]
                    [
                        IndexFilter(a; IdxaAbb; {$ a.speciale = 1 $})
                    ]
                ]
            ]
        ]
    ]
]
\end{forest}

Si faccia riferimento ai piani di accesso precedenti per il mancato uso di IdxuUtente e IdxsAbb.
\end{document}