\documentclass[11pt, italian, openany]{book}
% Set page margins
\usepackage[margin=2cm]{geometry}

\usepackage[]{graphicx}
\usepackage{setspace}
\usepackage{mathptmx}
\singlespace % interlinea singola
% If multiple images are to be added, a folder (path) with all the images can be added here 
\graphicspath{ {images/} }

\usepackage{hyperref}
\hypersetup{
    colorlinks=true,
    linkcolor=blue,
    filecolor=magenta,      
    urlcolor=blue,
}
 
% All page numbers positioned at the bottom of the page
\usepackage{fancyhdr}
\fancyhf{} % clear all header and footers
\fancyfoot[C]{\thepage}
\renewcommand{\headrulewidth}{0pt} % remove the header rule
\pagestyle{fancy}

% Changes the style of chapter headings
\usepackage{titlesec}

\titleformat{\chapter}
   {\normalfont\LARGE\bfseries}{\thechapter.}{1em}{}

% Change distance between chapter header and text
\titlespacing{\chapter}{0pt}{35pt}{\baselineskip}
\usepackage{titlesec}
\titleformat{\section}
  [hang] % \textlessshape\textgreater
  {\normalfont\bfseries\Large} % \textlessformat\textgreater
  {} % \textlesslabel\textgreater
  {0pt} % \textlesssep\textgreater
  {} % \textlessbefore code\textgreater
\renewcommand{\thesection}{} % Remove section references...
\renewcommand{\thesection}{\arabic{section}} %... from sections
\usepackage{titlesec}

\setcounter{secnumdepth}{3}

% Prevents LaTeX from filling out a page to the bottom
\raggedbottom

\usepackage{color}
\usepackage{xcolor}
\usepackage{enumitem}
\usepackage{amsmath}
\usepackage{subcaption}
% Code Listings
\definecolor{vgreen}{RGB}{104,180,104}
\definecolor{vblue}{RGB}{49,49,255}
\definecolor{vorange}{RGB}{255,143,102}
\definecolor{vlightgrey}{RGB}{245,245,245}

\definecolor{codegreen}{rgb}{0,0.6,0}
\definecolor{codegray}{rgb}{0.5,0.5,0.5}
\definecolor{codepurple}{rgb}{0.58,0,0.82}
\definecolor{backcolour}{rgb}{0.95,0.95,0.92}

\usepackage{listings}

\lstdefinestyle{code}{
    language=bash,
    backgroundcolor=\color{backcolour},   
    commentstyle=\color{codegreen},
    keywordstyle=\color{magenta},
    numberstyle=\tiny\color{codegray},
    stringstyle=\color{codepurple},
    basicstyle=\ttfamily\footnotesize,
    breakatwhitespace=false,         
    breaklines=true,                 
    captionpos=b,                    
    keepspaces=true,                 
    numbers=left,                    
    numbersep=5pt,                  
    showspaces=false,                
    showstringspaces=false,
    showtabs=false,                  
    tabsize=2
}

\begin{document}

\begin{sloppypar}
\begin{titlepage}
	\clearpage\thispagestyle{empty}
	\centering
	\vspace{1cm}

    \includegraphics[scale=0.60]{unipi-logo.png}
    
	{\normalsize \noindent Dipartimento di Informatica \\
	             Corso di Laurea in Informatica \par}
	
	\vspace{2cm}
	{\Huge \textbf{Reti di Calcolatori} \par }

    \vspace{4cm}

	{\normalsize Giacomo Trapani \\ Anno Accademico 2020/2021\par}

	\pagebreak

\end{titlepage}
\section*{Premessa.}
Si danno le seguenti definizioni:
\begin{itemize}[topsep=0pt]
\itemsep-0.3em
	\item \textbf{Rete}: interconnessione di dispositivi in grado di scambiarsi informazioni, quali sistemi terminali, router, switch e modem.
	\item \textbf{Host} (\textbf{sistema terminale}): sono macchine appartenenti agli utenti finali, dedicate all'esecuzione di applicazioni (e.g. computer,
	tablet, smartphone) oppure ``server" che forniscono servizi a diverse applicazioni utente (e.g. posta elettronica, web).
	\item \textbf{Dispositivi di interconnessione}: si distinguono in ``router" (i.e. dispositivi che interconnettono reti) e ``switch"
	(i.e. dispositivi che collegano fra loro molteplici host a livello locale).
	\item \textbf{Collegamenti} (\textbf{link}): mezzi trasmissivi (cablati, wireless).
\end{itemize}
\section{Tipi di Reti.}
\subsection*{LAN (Local Area Network).}
Le LAN vengono definite come reti di computer circoscritte a un'area geograficamente limitata (si estendono al pi\`u per alcuni chilometri,
sono circoscritte ad aree limitate come un ufficio, una scuola, un edificio etc.). Si cita Ethernet come tecnologia LAN popolare.

Le LAN sono tipicamente di propriet\`a di una qualche organizzazione (sono dunque reti private), connettono sistemi terminali (stampanti, PC,
workstations) mediante cavi di rame o connessioni wireless.

Si distinguono LAN con cavo condiviso (obsolete) e LAN con switch (moderne); il maggiore vantaggio della seconda sulla prima \`e che non ha
bisogno di far passare un pacchetto da ogni dispositivo connesso nel percorso dal router al (dispositivo) target.

\begin{figure}[h!]
	\begin{subfigure}{0.49 \linewidth} \centering
		\includegraphics[scale=0.35]{images/lan-obsoleta.png}
		\caption{LAN con cavo convidiso.}
	\end{subfigure}
	\begin{subfigure}{0.49 \linewidth} \centering
		\includegraphics[scale=0.35]{images/lan-moderna.png}
		\caption{LAN con switch.}
	\end{subfigure}
\end{figure}

\subsection*{WAN (Wide Area Network).}
Una WAN (o rete geografica) \`e una rete il cui compito \`e di interconnettere LAN o singoli host separati da distanze geografiche. Viene
tipicamente gestita da un operatore di rete che fornisce servizi ai clienti. Un esempio di WAN \`e la rete GARR.

\begin{figure}[h!]
	\begin{subfigure}{0.49 \linewidth} \centering
		\includegraphics[width=80mm,height=10mm]{images/wan-puntopunto.png}
		\caption{WAN punto-punto.}
	\end{subfigure}
	\begin{subfigure}{0.49 \linewidth} \centering
		\includegraphics[scale=0.25]{images/wan-commutazione.png}
		\caption{WAN a commutazione.}
	\end{subfigure}
\end{figure}

Si distinguono WAN punto-punto (collegano direttamente due dispositivi attraverso un mezzo trasmissivo (e.g. cavo in fibra ottica, ponti radio))
e WAN a commutazione (collegano pi\`u di due punti di terminazioni (e.g. dorsali Internet)); le seconde sono dotate di elementi di commutazione
(i.e. elaboratori specializzati utilizzati per connettere tra loro pi\`u linee di trasmissione).

\section{Tecniche di commutazione.}
Si definiscono ``tecniche di commutazione" le modalit\`a con cui viene determinato il percorso sorgente-destinazione e vengono dedicate ad esso le
risorse della rete. Si distinguono due tecniche: circuit switching (si parla di ``reti a commutazione di circuito") e packet switching (``reti a
commutazione di pacchetto").

\begin{figure}[h!]
	\begin{subfigure}{0.49 \linewidth} \centering
		\includegraphics[scale=0.25]{images/commutazione-circuito.png}
		\caption{Rete a commutazione di circuito.}
	\end{subfigure}
	\begin{subfigure}{0.49 \linewidth} \centering
		\includegraphics[scale=0.25]{images/commutazione-pacchetto.png}
		\caption{Rete a commutazione di pacchetto.}
	\end{subfigure}
\end{figure}

\subsection*{Circuit switching.}
La rete determina un percorso dalla sorgente alla destinazione, su questo viene riservato e garantito un rate di trasmissione costante per la
durata dell'intera sessione di comunicazione (una frazione della capacit\`a di trasmissione dei link).

Un esempio di rete a commutazione \`e la rete telefonica fissa tradizionale.

\textbf{Svantaggi}: le risorse rimangono inattive quando non utilizzate e non vengono mai condivise, necessaria una fase di instaurazione della
comunicazione (detta anche di ``setup") nella quale si configurano le tabelle di switching, la scarsa flessibilit\`a nell'uso delle risorse pu\`o
portare a ``overprovisioning" e/o a un sottoutilizzo di queste in presenza di rate di traffico variabile.

\textbf{Vantaggi}: performance garantita, overhead limitato, tecnologie di switching efficienti.

\subsection*{Packet switching.}
Il flusso di dati punto-punto viene suddiviso in pacchetti che condividono le risorse di rete e instradati singolarmente e indipendentemente
dagli altri. Si parla di trasmissione ``store and forward": il commutatore (tipicamente un router) deve aspettare di aver ricevuto per intero il
pacchetto prima di poter trasmettere sul collegamento in uscita.

Non c’\`e un canale dedicato e la sequenza dei pacchetti non segue uno schema prestabilito (si parla di ``multiplexing statistico"): i router
possono memorizzare i pacchetti nelle code (``buffer") nel caso in cui il collegamento sia gi\`a usato alla massima capacit\`a, introducendo
dunque dei ``ritardi di coda" e il rischio di ``congestione"; non si ha garanzia nelle prestazioni e si corre il rischio di avere una perdita
di pacchetti poich\'e i buffer hanno dimensione finita.

\textbf{Svantaggi}: alto overhead, tecnologie di inoltro non efficienti (si ha la necessit\`a di selezionare l’uscita per ogni pacchetto),
tempo di elaborazione ai router (si parla di ``routing table lookup"), accodamento ai router.

\textbf{Vantaggi}: risorse trasmissive usate solo su richiesta, segnalazione non necessaria.

\section{Internet.}
Si definisce ``internet" una rete costituita da due o pi\`u reti interconnesse; un esempio di internet \`e Internet, una rete a commutazione di
pacchetti composta da migliaia di reti interconnesse che seguono l'Internet Protocol e rispettano convenzioni precise per nomi e indirizzi, che
fornisce servizi di comunicazione alle applicazioni e per le applicazioni.

Le reti dei sistemi terminali sono connesse a Internet attraverso una gerarchia di fornitori di servizi internet (Internet Service Provider);
definisco ``dorsale" una ISP di primo livello. Si permette l'esistenza di ``Internet eXchange Point", definiti come punti di incontro per il peering
tra due o pi\`u ISP.

\begin{figure}[h!]
	\begin{subfigure}{0.49 \linewidth} \centering
		\includegraphics[scale=0.28]{images/internet-semplificata.png}
		\caption{Internet: rete di reti - versione semplificata.}
	\end{subfigure}
	\begin{subfigure}{0.49 \linewidth} \centering
		\includegraphics[scale=0.28]{images/internet-completa.png}
		\caption{Internet: rete di reti - versione moderna.}
	\end{subfigure}
\end{figure}

\subsection*{Metriche.}
Si definiscono i seguenti parametri, utilizzati per misurare le prestazioni della rete:
\begin{itemize}[topsep=0pt]
	\itemsep-0.3em
	\item \textbf{Larghezza di banda} (\textbf{Bandwidth}): larghezza dell’intervallo di frequenze utilizzato dal sistema trasmissivo misurato in Hertz.
	\item \textbf{Velocit\`a di trasmissione} (\textbf{Bitrate o transmission rate}): quantit\`a di dati (bits) che possono essere trasmessi (“inseriti
	nella linea”) nell'unit\`a di tempo (bits/secondo o bps) su un certo collegamento; dipende alla larghezza di banda e dalla tecnica trasmissiva
	utilizzata.
	\item \textbf{Throughput}: quantit\`a di dati che possono essere trasmessi da un nodo sorgente a un nodo destinazione in un certo intervallo di tempo
	al netto di perdite sulla rete, duplicazioni, protocolli etc.
	\item \textbf{Latenza}: tempo richiesto affinch\'e un messaggio arrivi a destinazione dal momento in cui il primo bit parte dalla sorgente; \`e
	definito come somma di ritardo di elaborazione, di accodamento, di trasmissione e di propagazione.
\end{itemize}
Si definiscono dunque i ritardi sopra elencati:
\begin{itemize}[topsep=0pt]
	\itemsep-0.3em
	\item \textbf{Ritardo di elaborazione}: \`e dovuto al controllo di errori sui bit e alla determinazione del canale di uscita.
	\item \textbf{Ritardo di accodamento}: \`e il tempo che un pacchetto spende all'interno del buffer (tipicamente di un router), \`e una
	quantit\`a aleatoria.
	\item \textbf{Ritardo di trasmissione}: \`e il tempo impiegato per trasmettere un pacchetto sul link; definendo L lunghezza del pacchetto in bit,
	R rate di trasmissione sul link in bps, vale \( \frac{L}{R} \).
	\item \textbf{Ritardo di propagazione}: \`e il tempo che 1bit impiega per essere propagato da un nodo all'altro; definendo d lunghezza del
	collegamento fisico, s velocit\`a di propagazione nel mezzo, vale \( \frac{d}{s} \).
\end{itemize}
Si definisce inoltre il ``ritardo end-to-end" come sommatoria del ritardo ai singoli nodi: \\\( R_{end-to-end} = \sum_{i=1}^{N} R_{nodo_i}\).

Si ricorda l'esistenza dei comandi ``traceroute", che traccia un pacchetto dal proprio dispositivo all'host mostrando anche il numero di passaggi
(salti) necessari per raggiungerlo assieme al ritardo dal mittente per ogni passaggio, e ``ping", che calcola il ritardo per la trasmissione dal
dispositivo corrente all'host.

\section{Modelli stratificati.}
Si definisce un ``protocollo" un insieme di regole che permettono a due entit\`a di comunicare; nei sistemi di comunicazione non \`e
sufficiente un singolo protocollo, si ricorre a una organizzazione di protocolli.

Si definisce inoltre ``aperto" un insieme di protocolli i cui dettagli sono disponibili pubblicamente e i cui cambiamenti sono gestiti da una
organizzazione la cui partecipazione \`e aperta al pubblico; definisco "sistema aperto" un sistema che implementa protocolli aperti.

Per la stratificazione si fa riferimento ai principi di ``separation of concern" (i.e. separazione degli interessi e delle responsabilit\`a, fare
ci\`o che compete, delegando ad altri tutto ci\`o che \`e delegabile) e ``information hiding" (i.e. nascondere tutte le informazioni che non sono
indispensabili per il committente per definire completamente un'operazione).

In concreto:
\begin{itemize}[topsep=0pt]
	\itemsep-0.3em
	\item Il modello permette di scomporre il problema in sottoproblemi pi\`u semplici il cui sviluppo \`e indipendente (da quello degli altri
	sottoproblemi).
	\item Ogni strato scambia informazioni con quelli adiacenti, ma comunica logicamente con il proprio omologo; fornisce informazioni allo strato
	superiore e usa i servizi offerti da quello inferiore.
	\item Un singolo strato svolge una sola funzione e realizza uno e un solo livello logico.
\end{itemize}

A questo punto l'obiettivo diventa realizzare una rete di calcolatori in cui qualsiasi terminale comunica con un qualsiasi fornitore di servizi
mediante qualsiasi rete; lo standard per l'interconnessione di sistemi aperti \`e il modello ISO/OSI.

\subsection*{Modello ISO/OSI.}
Si danno le seguenti definizioni nei termini del modello ISO/OSI:
\begin{itemize}[topsep=0pt]
	\itemsep-0.3em
	\item \textbf{Livello} (o \textbf{strato}): modulo interamente definito attraverso i servizi, protocolli e le interfacce che lo caratterizzano.
	\item \textbf{Servizio}: insieme di primitive (operazioni) che uno strato fornisce ad uno strato soprastante.
	\item \textbf{Interfaccia}: insieme di regole che governano il formato e il significato delle unit\`a di dati (es. messaggi, segmenti o
	pacchetti) che vengono scambiati tra due strati adiacenti della stessa entit\`a.
	\item \textbf{Protocollo}: insieme di regole che permettono a due entit\`a omologhe (stesso strato) uno scambio efficace ed efficiente delle
	informazioni, definiscono il formato e l’ordine dei messaggi inviati e ricevuti tra entit\`a omologhe della rete e le azioni che vengono
	fatte per la trasmissione e ricezione dei messaggi; occorre specificare sintassi e semantica di un messaggio e quali azioni intraprendere
	una volta ricevuto.
\end{itemize}

\begin{figure}[h!]
	\begin{subfigure}{0.49 \linewidth} \centering
		\includegraphics[scale=0.22]{images/stack-protocollare-iso_osi.png}
		\caption{Pila di protocolli.}
	\end{subfigure}
	\begin{subfigure}{0.49 \linewidth} \centering
		\includegraphics[scale=0.22]{images/gerarchia-layer-iso_osi.png}
		\caption{Gerarchia di strati (o "layer").}
	\end{subfigure}
\end{figure}

I layer dal livello 5 al 7 sono di supporto all'elaborazione e interazione con l'utente (vengono realizzati a livello software), gli altri 4
alla rete e all'infrastruttura trasmissiva (realizzati a livello hardware e software); tipicamente i nodi intermedi implementano solo i primi
4.

Il modello prevede inoltre l'incapsulamento: il flusso di informazioni - che ha origine al livello applicativo - discende verso i livelli inferiori e
viene progressivamente arricchita mediante l'aggiunta di headers.

Si danno in proposito le seguenti definizioni:
\begin{itemize}[topsep=0pt]
	\itemsep-0.3em
	\item \textbf{Header}: qualificazione del pacchetto dati al livello corrente.
	\item \textbf{Data}: payload proveniente dal livello precedente.
	\item \textbf{Trailer}: usato in funzione di trattamento dell'errore.
\end{itemize}

\subsection*{Stack protocollare TCP/IP.}
TCP/IP \`e una famiglia di protocolli attualmente utilizzata in Internet. Si tratta di una gerarchia di protocolli, ciascuno dei quali fornisce
funzionalità specifiche. Prende di riferimento il modello ISO/OSI e si articola nei seguenti 5 livelli (elencati dall'alto al basso):
\begin{itemize}[topsep=0pt]
	\itemsep-0.3em
	\item \textbf{Application Layer}: \`e di supporto alle applicazioni di rete, permette un collegamento logico end-to-end; si usa per lo scambio di
	pacchetti di informazioni detti ``messaggi". Include i protocolli HTTP, FTP, SMTP...
	\item \textbf{Transport Layer}: permette il trasporto di messaggi tra sistemi terminali; un pacchetto a questo livello prende il nome di ``segmento".
	Include i protocolli TCP (connection oriented, garantisce la consegna dei messaggi) e UDP (connectionless, non garantisce la consegna dei messaggi).
	\item \textbf{Network Layer}: si occupa dell'instradamento di pacchetti detti ``datagrammi" dalla sorgente alla destinazione tipicamente attraversando
	una serie di router. Include i protocolli IP, ICMP.
	\item \textbf{Link Layer}: si occupa di trasferire pacchetti detti ``frame" attraverso il collegamento tra elementi di rete vicini. Include i protocolli
	Ethernet, PPP, WiFi...
	\item \textbf{Physical Layer}: si occupa di trasferire individualmente i bit contenuti nei singoli frame da un nodo al successivo.
\end{itemize}

\begin{figure}[h!]
	\begin{subfigure}{0.49 \linewidth} \centering
		\includegraphics[scale=0.18]{images/livello-fisico-tcp_ip.png}
		\caption{Comunicazione in una internet - livello fisico.}
	\end{subfigure}
	\begin{subfigure}{0.49 \linewidth} \centering
		\includegraphics[scale=0.18]{images/incapsulamento-tcp_ip.png}
		\caption{Comunicazione in una internet - incapsulamento.}
	\end{subfigure}
\end{figure}

\subsubsection*{Application Layer.}
Si danno in merito le seguenti definizioni:
\begin{itemize}[topsep=0pt]
	\itemsep-0.3em
	\item \textbf{Application Programming Interface}: insieme di regole che un programmatore deve rispettare per l'utilizzo delle risorse.
	\item \textbf{Socket}: API che funge da interfaccia tra gli application e transport layer; due processi comunicano mandando dati alla socket (che
	fa da connessione a livello logico) il cui invio e ricezione sono responsabilit\`a dei restanti quattro livelli dello stack TCP/IP nel sistema
	operativo. \`E la API di Internet.
	\item \textbf{Uniform Resource Identifier}: stringa di caratteri che identifica una risorsa (i.e. qualsiasi cosa abbia una identit\`a) presente
	sulla rete. Se ne distinguono di due tipi:
	\vspace{-3.5mm}
	\begin{itemize}
		\itemsep-0.3em
		\item \textbf{Uniform Resource Locator}: sottoinsieme di URI che identifica le risorse attraverso il loro meccanismo di accesso. Segue la sintassi
		\textless{scheme}\textgreater{://}\textless{user}\textgreater{:}\textless{password}\textgreater{@}\textless{host}\textgreater{:}\textless{port}
		\textgreater{/}\textless{path}\textgreater:
		``scheme" identifica il protocollo, ``user" e ``password" sono opzionali e generalmente deprecati, ``host" \`e il nome di dominio di un host,
		``port" \`e il numero di porta del server, ``path" identifica la risorsa nel contesto specificato.
		\item \textbf{Uniform Resource Name}: sottoinsieme di URI che devono rimanere globalmente unici e persistente anche quando la risorsa cessa di
		esistere e diventa non disponibile.
	\end{itemize}
\end{itemize}

A questo livello si distinguono due modelli architetturali:
\begin{itemize}[topsep=0pt]
	\itemsep-0.3em
	\item \textbf{Client-Server}: esiste un host sempre attivo (il ``server") che serve le richieste di altri host (i ``clients").
	\item \textbf{Peer-to-Peer}: la comunicazione avviene tra coppie di dispositivi non sempre attivi (i ``peer") che possono offrire servizi e inviare
	richieste.
\end{itemize}

Per poter inviare dati risulta necessario poter indirizzare opportunamente il processo destinatario: per Internet si usa la coppia indirizzo IP (i.e. un
intero a 32bit che identifica univocamente una macchina host), numero di porta (un intero a 16bit che identifica univocamente un processo all'interno di
una macchina).

Il livello di trasporto mette a disposizione due protocolli diversi per la comunicazione:
\begin{itemize}[topsep=0pt]
	\itemsep-0.3em
	\item \textbf{TCP}: ha le seguenti caratteristiche:
	\vspace{-3.5mm}
	\begin{itemize}
		\itemsep-0.3em
		\item \textbf{Connection-oriented}: \`e necessaria una fase di setup iniziale tra client e server prima di poter trasmettere messaggi (si parla
		di ``handshaking").
		\item \textbf{Controllo del flusso}: il mittente non "inonder\`a" di dati il destinatario.
		\item \textbf{Controllo di congestione}: ``strangola" il mittente (client o server) quando la rete è sovraccarica.
		\item Garantisce il trasporto affidabile tra processo mittente e destinatario.
		\item Non offre garanzie di timing n\'e di ampiezza minima di banda.
	\end{itemize}
	\item \textbf{UDP}: ha le seguenti caratteristiche:
	\vspace{-3.5mm}
	\begin{itemize}
		\itemsep-0.3em
		\item \textbf{Connectionless}: non orientato alle connessioni, non prevede una fase di setup iniziale.
		\item Trasporto non affidabile.
		\item Non implementa meccanismi di controllo di flusso.
		\item Non implementa meccanismi per il controllo di congestione.
		\item Non offre garanzie di timing n\'e ampiezza minima di banda.
	\end{itemize}
\end{itemize}

\begin{figure}[h!]
	\begin{subfigure}{0.49 \linewidth} \centering
		\includegraphics[scale=0.18]{images/trasporto-applicazioni-comuni.png}
		\caption{Tipo di trasporto richiesto da applicazioni comuni.}
	\end{subfigure}
	\begin{subfigure}{0.49 \linewidth} \centering
		\includegraphics[scale=0.18]{images/applicazioni-internet.png}
		\caption{Esempi di applicazioni Internet e i relativi protocolli.}
	\end{subfigure}
\end{figure}

Si passa dunque alla trattazione dei protocolli implementati a questo livello:
\begin{itemize}[topsep=0pt]
	\itemsep-0.3em
	\item \textbf{HTTP}: ha le seguenti caratteristiche:
	\vspace{-3.5mm}
	\begin{itemize}
		\itemsep-0.3em
		\item \textbf{Stateless}: non ha memoria di stato: le coppie richiesta, risposta sono indipendenti; ogni richiesta viene eseguita indipendentemente
		da quelle che la han preceduta.
		\item \textbf{Richiesta/Risposta}: La connessione viene iniziata dal client, che invia un messaggio di richiesta detto ``request" a cui il server
		risponde con un messaggio di risposta detto ``response".
		\item \textbf{Content negotiation}: permette di selezionare la rappresentazione appropriata per una risorsa (e.g. formato, lingua etc) quando
		viene servita richiesta.
		\item \textbf{Web caching}: si memorizzano copie temporanee di risorse Web (es. pagine HTML, immagini) e vengono servite al client per ridurre
		l’uso di risorse (e.g. banda, workload sul server) e il tempo di risposta al client.
		\item Implementa i ``cookie": un pacchetto di informazioni assegnato dal server alla prima richiesta del client che, a sua volta, lo legge dal
		messaggio di risposta (in cui trova una linea ``set-cookie") e lo salva in un file; il cookie viene poi aggiunto dal client a tutte le successive
		richieste a quel sito (conterranno la linea ``cookie:\textless{cookie}\textgreater{)}".
		\item Usa TCP.
		\item HTTP/1.0 permetteva solo connessioni non persistenti, HTTP/1.1 prevede che - di default - le connessioni siano invece persistenti e vengano
		mantenute fino a una richiesta di chiusura. I server che rispettano HTTP/1.1 supportano inoltre il pipelining (i.e. consiste nell’invio da parte del
		client di molteplici richieste senza aspettare la ricezione di ciascuna risposta, il server deve inviare le risposte nello stesso ordine in cui
		sono state ricevute; riguarda solo metodi idempotenti); implementato in pochi browser, introduce il rischio di Head of Line Blocking (i.e. se una
		richiesta richiede tempo per essere processata le risposte a quelle successive sono bloccate).
	\end{itemize}

\pagebreak

	\begin{figure}[h!]
		\begin{subfigure}{0.49 \linewidth} \centering
			\includegraphics[scale=0.25]{images/http-request.png}
			\caption{HTTP Request Message.}
		\end{subfigure}
		\begin{subfigure}{0.49 \linewidth} \centering
			\includegraphics[scale=0.25]{images/http-response.png}
			\caption{HTTP Response Message.}
		\end{subfigure}
	\end{figure}

	I metodi messi a disposizione dal protocollo sono:
	\vspace{-3.5mm}
	\begin{itemize}
		\itemsep-0.3em
		\item \textbf{OPTIONS}: richiede informazioni sulle opzioni di comunicazione (i.e. quali metodi) sono associati a una URL o al server stesso.
		\item \textbf{GET}: richiede il trasferimento di una risorsa identificata da una URL o operazioni associate all’URL stessa; la semantica pu\`o
		essere modificata in un ``conditional GET" se l'header contiene uno dei campi ``If-Modified-Since", ``If-Unmodified-Since", ``If-Match",
		``If-None-Match" o ``If-Range", ``partial GET" se contiene il campo ``range".
		\item \textbf{HEAD}: identico al metodo GET con l'eccezione che la risposta non contiene il campo ``message-body".
		\item \textbf{POST}: serve per inviare dal client al server informazioni inserite nel body del messaggio, la funzione effettiva determinata \`e
		dal server e dipendente tipicamente dalla Request-URI; tipicamente non \`e abilitato sui server pubblici.
		\item \textbf{PUT}: il client chiede al server di creare/modificare una risorsa.
		\item \textbf{DELETE}: il client chiede di cancellare una risorsa.
		\item TRACE.
	\end{itemize}

	Si definiscono ``safe" i metodi che non modificano la rappresentazione di una risorsa (i.e. GET, HEAD, OPTIONS, TRACE), ``idempotent" quelli che godono
	della propriet\`a di idempotenza (i.e. GET, HEAD, PUT, DELETE, OPTIONS, TRACE).

	\item \textbf{TErminaL NETwork}: ha le seguenti caratteristiche
	\vspace{-3.5mm}
	\begin{itemize}
		\itemsep-0.3em
		\item Prevede un programma server che accetta le richieste e un programma client che effettua le richieste; quest'ultimo interagisce con il terminale
		utente sull'host locale e scambia messaggi con il Telnet server.
		\item Usa TCP e persiste per l'intera durata della sessione, il client si connette alla porta 23 del server.
		\item Il client accetta le battute di tasti del terminale e le invia al server nel quale un software chiamato ``pseudo terminal driver" fa da entry point
		del sistema operativo: consente di trasferire caratteri a un processo come se provenissero dal terminale.
		Il client accetta i caratteri che il server manda indietro e li visualizza sul terminale utente.
		\item  I messaggi di controllo iniziali sono usati per scambiare informazioni sulle caratteristiche degli host (si parla di ``Telnet option negotiation").
		\item Il client invia (in chiaro) un login identifier e una password (non \`e sicuro, \`e stato sostituito da ``SSH").
	\end{itemize}

	\begin{figure}[h!]
		\begin{subfigure}{0.49 \linewidth} \centering
			\includegraphics[scale=0.25]{images/telnet.png}
			\caption{TELNET: Login da remoto.}
		\end{subfigure}
		\begin{subfigure}{0.49 \linewidth} \centering
			\includegraphics[scale=0.25]{images/telnet-nvt.png}
			\caption{TELNET: NVT.}
		\end{subfigure}
	\end{figure}

	Il terminale in esecuzione sia sul client sia sul server \`e un ``Network Virtual Terminal": gli host traducono le loro caratteristiche locali cos\`i da
	apparire esternamente come un NVT e assumono che l’host remoto sia un NVT; si scambiano dati e comandi in formato 7-bit US-ASCII sullo stesso canale (prevede
	dunque ``in-band signaling"), ogni carattere \`e inviato come un ottetto con il primo bit settato a 0 se si tratta di un dato, a 1 per i comandi.

	\item \textbf{Simple Mail Transfer Protocol}:

	\item \textbf{File Transfer Protocol}: ha le seguenti caratteristiche:
	\vspace{-3.5mm}
	\begin{itemize}
		\itemsep-0.3em
		\item Permette control connection (i.e scambio di comandi e risposte tra client e server). Segue il protocollo Telnet.

		Il client FTP contatta il server FTP alla porta 21, ottiene l’autorizzazione sulla connessione di controllo e invia i comandi su di questa
		(e.g. cambio directory, invio file, ecc.). La connessione di controllo \`e persistente
		\item Permette data connection (i.e. connessione su cui i dati sono trasferiti con modi e tipi specificati). I dati trasferiti possono essere
		parte di un file, un file o un set di file.

		Il server apre una connessione dati TCP con il client (caveat: non \`e il client a farlo, si parla di ``modalit\`a attiva"), trasferisce il
		file sulla connessione dati e, terminata l'operazione, il server chiude la connessione. La connessione dati non \`e persistente, ne viene
		aperta e chiusa una per ciascun trasferimento.
		\item Si usa il protocollo di trasporto TCP, per cui si permettono due differenti modalit\`a: quella ``attiva" (descritta precedentemente,
		usata di default) e quella ``passiva" (in questo caso il client chiede al server di mettersi in ascolto su una porta per una connessione dati,
		ottiene questo numero di porta - tipicamente 20 - e lo usa per aprire la connessione con il server).
		\item \`E un protocollo stateful.
		\item Fa uso di codici di ritorno.
		\item Il collegamento pu\`o essere anonimo (si parla di ``anonymous FTP") o reso sicuro con TLS.
	\end{itemize}
	I comandi messi a disposizione sono i seguenti:
	\vspace{-3.5mm}
	\begin{itemize}
		\itemsep -0.3em
		\item \textbf{USER} \textless{username}\textgreater{.}
		\item \textbf{PASS} \textless{password}\textgreater{.}
		\item \textbf{LIST}: elenca i file della directory corrente.
		\item \textbf{NLST}: richiede elenco file e directory.
		\item \textbf{RETR} \textless{filename}\textgreater{:} recupera filename dalla directory corrente.
		\item \textbf{STOR} \textless{filename}\textgreater{:} memorizza (PUT) filename nell'host remoto.
		\item \textbf{ABOR}: interrompe l’ultimo comando ed i trasferimenti in corso.
		\item \textbf{PORT}: indirizzo e numero di porta del client.
		\item \textbf{SYST}: il server restituisce il tipo di sistema.
		\item \textbf{QUIT}: chiude la connessione.
	\end{itemize}
\end{itemize}

\pagebreak

\subsection*{Formato dei messaggi.}
Appare evidente che il formato ASCII 7bit non permetta di rappresentare caratteri speciali: si introduce il protocollo \textbf{Multipurpose
Internet Mail Extension} (MIME). 

Vengono fornite delle regole di codifica e di decodifica per trasformare caratteri speciali e contenuti multimediali in caratteri ASCII 7bit;
questo approccio ha permesso di inviare messaggi MIME usando protocolli e mail server esistenti (risulta necessario modificare gli User Agents).

I mail server moderni possono negoziare l'invio di dati in codifica binaria (8 bit); se non ha successo, inviano caratteri ASCII 7bit seguendo
il protocollo MIME.

\subsection*{Protocolli di accessi alla Mail.}
SMTP \`e un protocollo di tipo "push", per la ricezione risulta necessario un protocollo di tipo "pull". Alcuni esempi sono: Post Office Protocol
(POP), Internet Mail Access Protocol (IMAP), HTTP (quando User Agent è un browser).

\section*{Strato Applicativo DNS.}
Si danno le seguenti definizioni:

\textbf{Nome}: identifica un oggetto con una sequenza di caratteri scelti da un alfabeto finito; si usa per motivi mnemonici e per disaccoppiare
il livello applicativo da quello di rete.

\textbf{Indirizzo}: identifica la locazione dell'oggetto (i dispositivi connessi alla rete vengono individuati mediante indirizzi da 4byte detti
"indirizzi IP").

\textbf{Dominio}: sottoalbero nello spazio dei nomi di dominio che viene identificato dal nome di dominio del nodo in cima al sottoalbero.


Inizialmente l'associazione tra nomi logici e indirizzi IP era statica (i.e. tutti i nomi logici e i relativi indirizzi IP erano contenuti in
un file detto "host file"); questo approccio risulta attualmente impraticabile. Si ricorre al sistema DNS.

\subsection*{Domain Name System (DNS).}
Il DNS si trova al livello applicativo, viene fornito da sistemi terminali e usa servizi del livello di trasporto per trasferire messaggi,
non interagisce direttamente con gli utenti a meno che non ci siano errori.

E' un meccanismo che deve specificare la sintassi dei nomi e le regole per gestirli, consentire la conversione da nomi a indirizzi e viceversa.
è costituito essenzialmente da:
\begin{itemize}[topsep=0pt]
	\itemsep-0.3em
	\item uno schema di assegnazione dei nomi gerarchico e basato su domini;
	\item un database distribuito contenente i nomi e le corrispondenze con gli indirizzi IP implementato con una gerarchia di name server;
	\item un protocollo per la distribuzione delle informazioni sui nomi tra name server host, router, name server comunicano per risolvere nomi
	(traduzione nome/indirizzo) utilizzando tipicamente UDP con porta 53.
\end{itemize}

Offre i seguenti servizi:
\begin{itemize}[topsep=0pt]
	\itemsep-0.3em
	\item risoluzione di nomi di alto livello (hostname) in indirizzi IP;
	\item host aliasing (i.e. un host può avere più nomi, vengono tradotti nel nome canonico e - di seguito - nell'indirizzo IP);
	\item mail server aliasing (i.e. sinonimi per mail server; e.g. nome identico per mail server e web server);
	\item distribuzione carico (i.e. un hostname canonico può corrispondere a più indirizzi IP, il DNS restituisce la lista variandone l'ordine per ogni risposta).
\end{itemize}

Per i nomi viene usata una struttura gerarchica (i.e. un nome è costituito da diverse parti (e.g. lab3.di.unipi.it)); in questo modo
l'assegnazione dei nomi può essere delegata a singole organizzazioni e la responsabilità di convertire nomi e indirizzi distribuita.

\subsection*{Name Server.}
Si definisce come un programma che gestisce la conversione da nome di dominio a indirizzo IP.

Si definiscono inoltre:

\textbf{Server Radice}: responsabile dei record della zona radice e restituisce le informazioni sui name server di TLD.

\textbf{Server Top-Level Domain}: mantiene le informazioni dei nomi di dominio che appartengono a un certo TLD, restituisce le informazioni
sui name server di competenza dei sottodomini.

\textbf{Server di Competenza} (Authoritative Name Server): autorità per una certa zona, memorizza nome e indirizzo IP di un insieme di host dei
quali può effettuare traduzioni nome/indirizzo; per una certa zona ci possono essere server di competenza primari e secondari:
quelli primari mantengono il file di zona, quelli secondari ricevono il file di zona e offrono il servizio di risoluzione.

\textbf{Local Name Servers}: non appartengono strettamente alla gerarchia dei server, ogni ISP (università, società, ISP) ha il proprio; a questi
vengono inizialmente rivolte le query DNS: operano da proxy e inoltrano la query in una gerarchia di server DNS.

Quando un programma (es. browser) deve trasformare un nome in
un indirizzo IP chiama un programma detto resolver, passando il
nome come parametro di ingresso.
Il resolver interroga il local name server, che cerca il nome nelle sue
tabelle e restituisce l’indirizzo al resolver oppure inoltra la query
alla gerarchia DNS.

INSERIRE IMMAGINI QUERY RICORSIVA, QUERY ITERATIVA.

Una volta appresa una associazione, questa viene salvata su una cache i cui record vengono cancellati dopo un certo tempo detto ``timeout-ttl".

I record sono una quadrupla definita come ``name", ``value", ``type", ``ttl".

Dal parametro ``type" dipende il significato della coppia ``name" e ``value"; si citano alcuni esempi:
\begin{itemize}[topsep=0pt]
	\itemsep-0.3em
	\item \textbf{A}: `name" sta per hostname, value per indirizzo IP.
	\item \textbf{CNAME}: `name" sta per hostname (o un suo sinonimo), ``value" per il nome canonico dell'host.
	\item \textbf{NS}: ``name" sta per nome di dominio, ``value" per l'authorative name server per quel dominio.
	\item \textbf{MX}: ``name" sta per nome di dominio, ``value" per il nome canonico del server di posta associato a quel dominio.
\end{itemize}

Il protocollo DNS ha le seguenti caratteristiche:
\begin{itemize}[topsep=0pt]
	\itemsep-0.3em
	\item Usa connessione UDP sulla porta 53.
	\item Permette uso di TCP (tipicamente usata per messaggi di grandi dimensioni tra server DNS)
	\item Messaggi di query e reply hanno lo stesso formato.
	\item Ha un header formato dai seguenti campi:
	\vspace{-3.5mm}
	\begin{itemize}
		\itemsep-0.3em
		\item identification: intero a 16bit uguale per la coppia query, reply.
		\item flags: 0 se query, 1 se risposta.
		\item autorevolezza della risposta.
	\end{itemize}
	\item Domande: campi per il nome richiesto e il tipo di query
	\item Risposte: RR nella risposta alla domanda
	\item Competenza: record relative ai server di competenza
	\item Informazioni aggiuntive
\end{itemize}

INSERIRE IMMAGINE MESSAGGI DNS.

Per interrogazioni a server DNS si usa il tool "dig".

\pagebreak
\subsubsection*{Transport Layer.}
Si danno le seguenti definizioni:
\begin{itemize}[topsep=0pt]
	\itemsep-0.3em
	\item \textbf{Multiplexing}: provvede all'``accorpamento” dei flussi dati dai processi verso la rete, li ``imbusta" con un preambolo. \`E
	basato sui socket address dei processi.
	\item \textbf{Demultiplexing}: provvede allo ``smistamento” dei pacchetti fra la rete e le applicazioni (i.e. dirige i dati ricevuti al
	processo corretto). \`E basato sui socket address dei processi.
	\item \textbf{Porta}: intero senza segno a 16bit assegnato a un punto di demultiplexing dei protocolli TCP o UDP. Si distinguono tre differenti
	categorie di porte: ``system ports" (da 0 a 1023, identificano processi server, assegnate da IANA), ``user ports" (da 1024 a 49151, assegnate
	da IANA), ``dynamic ports" (le restanti, non assegnate da IANA).
	\item \textbf{Indirizzo IP}: indirizzo di 32bit presente nello stack TCP/IP.
	\item \textbf{Socket address}: identificativo formato dalla coppia ``porta", ``indirizzo IP".
	\item \textbf{Socket TCP}: identificativo formato dalla quadrupla ``indirizzo IP di origine", ``numero di porta di origine", ``indirizzo IP di
	destinazione", ``numero di porta di destinazione".
\end{itemize}

Offre i seguenti servizi:
\begin{itemize}[topsep=0pt]
	\itemsep-0.3em
	\item \textbf{Multiplexing}.
	\item \textbf{Demultiplexing}: se ne distinguono due tipi:
	\vspace{-3.5mm}
	\begin{itemize}
		\itemsep-0.3em
		\item \textbf{senza connessione} (\textbf{UDP}): lo strato di trasporto dell'host ricevente consegna il segmento UDP alla socket
		identificata dal socket address destinazione; i datagrammi inviati da mittenti differenti e con stessa destinazione vengono consegnati
		alla stessa socket.
		\item \textbf{con connessione} (\textbf{TCP}): lo strato di trasporto dell'host ricevente usa i parametri della socket TCP per inviare il
		segmento alla socket appropriata (e.g. un server Web pu\`o creare una socket per ogni client connesso); un host server pu\`o supportare
		pi\`u socket alla volta.
	\end{itemize}

	\begin{figure}[h!]
		\begin{subfigure}{0.49 \linewidth} \centering
			\includegraphics[scale=0.22]{images/demultiplexing-senza-connessione.png}
			\caption{Demultiplexing senza connessione.}
		\end{subfigure}
		\begin{subfigure}{0.49 \linewidth} \centering
			\includegraphics[scale=0.22]{images/demultiplexing-orientato-alla-connessione.png}
			\caption{Demultiplexing orientato alla connessione.}
		\end{subfigure}
	\end{figure}
	
	\item \textbf{Protocollo TCP}: ha le seguenti caratteristiche:
	\vspace{-3.5mm}
	\begin{itemize}
		\itemsep-0.3em
		\item \textbf{Orientato allo stream}: i dati sono un flusso di byte ordinati ma non strutturati; numera i byte (NON i segmenti). Si
		definiscono in merito ``numero di sequenza" associato a un segmento il numero (nel flusso) del primo byte del segmento e ``numero di
		riscontro" il numero dell'ultimo byte correttamente ricevuto + 1.
		\item \textbf{Orientato alla connessione}: i processi effettuano un handshake (i.e. informazioni preliminari per preparare lo scambio
		dei dati) prima dello scambio dei dati; si definisce orientato poich\'e lo stato della connessione risiede sui punti terminali e non
		sugli elementi intermedi della rete.
		\item \textbf{Connessione full-duplex}: il flusso dati tra due host pu\`o avvenire contemporaneamente nelle due direzioni.
		\item Garantisce trasferimento dati ordinato e affidabile (i.e. corregge eventuali errori).
		\item Implementa multiplexing/demultiplexing.
		\item Implementa meccanismi di controllo di flusso (i.e. evita di spedire pi\`u dati di quanti il ricevitore possa elaborare).
		\item Implementa meccanismi di controllo di congestione (i.e. gestisce opportunamente situazioni di sovraccarico della rete).
		\item Suddivide il flusso di byte in segmenti in modo indipendente dal programma applicativo che li ha generati (si parla di ``trasferimento
		bufferizzato".
	\end{itemize}
\pagebreak

	\begin{figure}[h!]
		\begin{subfigure}{0.49 \linewidth} \centering
			\includegraphics[scale=0.25]{images/tcp-trasferimento-bufferizzato.png}
			\caption{Trasferimento bufferizzato.}
		\end{subfigure}
		\begin{subfigure}{0.49 \linewidth} \centering
			\includegraphics[scale=0.25]{images/tcp-formato-segmenti.png}
			\caption{Formato dei segmenti TCP.}
		\end{subfigure}
	\end{figure}

	Un segmento TCP permette ``selective acknowledgement" (nel campo ``Opzioni", non supportato da tutte le implementazioni) che permette di
	specificare quali segmenti sono stati ricevuti, e supporta i seguenti flags (in grassetto i pi\`u importanti):
	\vspace{-3.5mm}
	\begin{itemize}
		\itemsep-0.3em
		\item URG: Il campo Puntatore Urgente contiene dati significativi da trasferire in via prioritaria.
		\item \textbf{ACK}: Il campo Numero di Riscontro contiene dati significativi
		\item PSH: Funzione Push (trasferimento immediato dei dati in un segmento dal trasporto al livello applicativo).
		\item RST: Reset della connessione, implica una chiusura forzata della connessione.
		\item \textbf{SYN}: Sincronizza il Numero di Sequenza. Messo a 1 durante l'handshake.
		\item \textbf{FIN}: Non ci sono altri dati dal mittente, implica la chiusura della connessione.
	\end{itemize}

	\item \textbf{Protocollo UDP}:
\end{itemize}

\end{sloppypar}
\end{document}