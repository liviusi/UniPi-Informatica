\documentclass[10pt, italian, openany, landscape]{book}
% Set page margins
\usepackage[top=100pt,bottom=100pt,left=68pt,right=66pt]{geometry}

\usepackage[]{graphicx}
% If multiple images are to be added, a folder (path) with all the images can be added here 
\graphicspath{ {images/} }

\usepackage{hyperref}
\hypersetup{
    colorlinks=true,
    linkcolor=blue,
    filecolor=magenta,      
    urlcolor=blue,
}
 
% All page numbers positioned at the bottom of the page
\usepackage{fancyhdr}
\usepackage{lscape}
\usepackage{adjustbox}

\newcommand*{\MyIncludeGraphics}[2][]{%
\begin{adjustbox}{max size={0.8\textwidth}{\textheight}}
    \includegraphics[#1]{#2}%
\end{adjustbox}
}

\fancyhf{} % clear all header and footers
\fancyfoot[C]{\thepage}
\renewcommand{\headrulewidth}{0pt} % remove the header rule
\pagestyle{fancy}

% Changes the style of chapter headings
\usepackage{titlesec}

\titleformat{\chapter}
   {\normalfont\LARGE\bfseries}{\thechapter.}{1em}{}

% Change distance between chapter header and text
\titlespacing{\chapter}{0pt}{50pt}{2\baselineskip}
\usepackage{titlesec}
\titleformat{\section}
  [hang] % <shape>
  {\normalfont\bfseries\Large} % <format>
  {} % <label>
  {0pt} % <sep>
  {} % <before code>
\renewcommand{\thesection}{} % Remove section references...
\renewcommand{\thesubsection}{\arabic{subsection}} %... from subsections

% Numbered subsections
\setcounter{secnumdepth}{3}

% Prevents LaTeX from filling out a page to the bottom
\raggedbottom

\usepackage{listings}
\usepackage{color}
\usepackage{xcolor}
\usepackage{amsmath}
% Code Listings
\definecolor{vgreen}{RGB}{104,180,104}
\definecolor{vblue}{RGB}{49,49,255}
\definecolor{vorange}{RGB}{255,143,102}
\definecolor{vlightgrey}{RGB}{245,245,245}
\lstdefinestyle{bash} {
	language=bash,
	basicstyle=\ttfamily,
	keywordstyle=\color{vblue},
	identifierstyle=\color{black},
	commentstyle=\color{vgreen},
	tabsize=4,
	backgroundcolor = \color{vlightgrey},
	literate=*{:}{:}1
}

\definecolor{codegreen}{rgb}{0,0.6,0}
\definecolor{codegray}{rgb}{0.5,0.5,0.5}
\definecolor{codepurple}{rgb}{0.58,0,0.82}
\definecolor{backcolour}{rgb}{0.95,0.95,0.92}

\lstdefinestyle{codeStyle}{
	language=Java,
    backgroundcolor=\color{backcolour},   
    commentstyle=\color{codegreen},
    keywordstyle=\color{magenta},
    numberstyle=\tiny\color{codegray},
    stringstyle=\color{codepurple},
    basicstyle=\ttfamily\footnotesize,
    breakatwhitespace=false,         
    breaklines=true,                 
    captionpos=b,                    
    keepspaces=true,                 
    numbers=left,                    
    numbersep=5pt,                  
    showspaces=false,                
    showstringspaces=false,
    showtabs=false,                  
    tabsize=2
}

\begin{document}

\begin{titlepage}
	\clearpage\thispagestyle{empty}
	\centering
	\vspace{1cm}

    \includegraphics[scale=0.60]{unipi-logo.png}
    
	{\normalsize \noindent Dipartimento di Informatica \\
	             Corso di Laurea in Informatica \par}
	
	\vspace{2cm}
	{\Huge \textbf{Un giorno al Museo} \par}
	\vspace{1cm}
	{\large Ingegneria del Software a.a. 2020/2021}

    \begin{minipage}[t]{0.47\textwidth}
    	{\large{ Prof.ssa Roberta Gori}}
    \end{minipage}\hfill\begin{minipage}[t]{0.47\textwidth}\raggedleft
        {\large {S. C. \\ ---- - ----\\ }}
        {\large {D. I. \\ ---- - ----\\ }}
    	{\large {Giacomo Trapani \\ ---- - ----\\ }}
    \end{minipage}

    \vspace{4cm}

	\pagebreak

\end{titlepage}

% A * after the section/chapter command indicates an unnumbered header which will not be added to the table of contents

\section{Esercizio 1}
Si assume che un utente possa possedere al massimo \textbf{uno e un solo abbonamento} alla volta.

% \MyIncludeGraphics{esercizio1.png}

\pagebreak
\section{Esercizio2}
Come si evince dal testo, all'interno di un museo esiste \textbf{una e una sola Biglietteria}, risultano \textbf{opzionali}
le presenze di \textbf{Bar}, \textbf{Ristorante} (presenti al pi\`u \textbf{una e una sola volta} ciascuno), \textbf{Sale} e \textbf{Bagni} (presenti invece
in quantit\`a non meglio precisate).


% \MyIncludeGraphics{esercizio2.png}

\pagebreak

\section{Esercizio 3}
Si ricorda che il frammento di piantina del museo da rappresentare \textbf{non} costituisce una istanza completa
o corretta di un museo: nello specifico, manca un \textbf{Ambiente} connesso al \textbf{Varco 33}.

% \MyIncludeGraphics{esercizio3.png}


\pagebreak

\section{Esercizio 4}
Scegliendo di modellizzare la scelta del modulo a cui un dipendente del museo viene assegnato come
un segnale ricevuto dall'esterno ed essendo presente un vincolo sulla \textbf{durata} dei singoli moduli, si sceglie
di eliminare la logica (della durata) temporale dal diagramma delle attivit\`a di \textbf{Sala} e di \textbf{Reception} (che vengono
dunque modellizzati come un ciclo infinito) e di forzare la terminazione di questi al livello superiore (ossia al livello di \textbf{Modulo}.)
In questo modo l'attivit\`a \textbf{Turno} non \`e altro che una sequenza di 3 attivit\`a modulo.

% \MyIncludeGraphics{esercizio4.png}

\pagebreak

\section{Esercizio 5}
Si assume che la vita di un dispositivo non termini n\'e alla chiusura del museo n\'e quando questo si scarichi
(questa infatti termina se e solo se il dispositivo risulta \textbf{rotto}) e che questi partano scarichi in modo tale
da uniformarli.

% \MyIncludeGraphics{esercizio5.png}
\pagebreak

\section{Esercizio 6}
Si sceglie di dividere il sistema software Un giorno al Museo in due grosse componenti: \textbf{Gestione musei} e 
\textbf{Gestione Visite}. Si assume che la prima gestisca il database delle audioguide a cui accede ogni dispositivo
elettronico nel momento in cui un visitatore inquadra un codice QR per avere a disposizione - appunto - l'audioguida
dell'opera scelta. Dal testo si deduce inoltre che i varchi comunicano direttamente col dispositivo elettronico scrivendo in
questo i dati riguardo il transito in una determinata sala e che questi vengano elaborati dalla seconda componente menzionata
(poich\'e a disposizione di quest'ultima viene messa una operazione di lettura dei dati dai dispositivi elettronici).

% \MyIncludeGraphics{esercizio6.png}

\pagebreak

\section{Esercizio 7}
Si assume che il software Un giorno al Museo fornisca a ogni museo appartenente al sistema museale un eseguibile
che comunica col \textbf{server centrale} sul quale sono salvati sia gli abbonamenti sia i dati dei singoli musei. Si assume inoltre
che la gestione del database delle audioguide non sia affidata al sistema centrale, ma sia responsabilit\`a dei
singoli musei e che questo comunichi coi dispositivi elettronici.


% \MyIncludeGraphics{esercizio7.png}
\pagebreak

\section{Esercizio 8}
Si assume che per il calcolo delle tariffe e per il passaggio da
tempo assoluto (che si assume essere il formato adottato dai dati inviati dai varchi)
a \textbf{tempo relativo} (ossia il formato che verr\`a utilizzato nei punti successivi)
sia necessario un \textbf{ProxyOrologio}.

% \MyIncludeGraphics{esercizio8.png}
\pagebreak

\section{Esercizio 9}
\subsection{Esercizio 9.a}
Assumiamo nello svolgimento dell’esercizio che:
\begin{itemize}
    \item l’identificatore di ogni sala sia un intero (tipicamente una chiave primaria nel database delle sale);
    \item i metodi \textit{bool salaPermanente(int ID)} e \textit{bool salaTemporanea(int ID)} internamente usino un oggetto di tipo che contiene gli attributi corrispondenti presenti nella sala e che usino uno stub che simuli la connessione al database per recuperare le informazioni necessarie.
\end{itemize} Un possibile stub è il seguente:
\begin{lstlisting}[style=codeStyle]
    bool salaPermanente(int ID)
    {
        return ID % 3 == 0;
    }

    bool salaTemporanea(int ID)
    {
        return ID % 3 == 1;
    }
\end{lstlisting}
che rispetta le specifiche del sistema in quanto non permette di avere sale contemporaneamente temporanee e permanenti.
Si definiscono inoltre le due classi di equivalenza:
\[ errore = \{ [Passaggio] \ lp \ | \ (lp = NULL) \ \lor \ (\exists \ int \ i \ \in [0; lp.length()] \ | \ lp[i] = NULL)) \}. \]
\begin{gather*}
(\forall (h, k) \ | \ k + h \ \in [0; lp.length() - 1] \ \wedge h*k \geq 0. \ \mathbf{X}_{h, k} = \{ [Passaggio] \ lp \ | \ (\exists \ I = \{i_1, ..., i_k\} \subseteq [1, ..., lp.length()-1] \ | \\
(\forall i \in I. \ lp[i].orario - lp[i-1].orario \geq 30 \ \wedge \ salaPermanente(lp[i-1].sala)) \ \wedge \\
(\exists J = \{j_1, ..., j_h\} \subseteq [1, ..., lp-length()-1] \ | \ (\forall j \in J. lp[j].orario - lp[j-1].orario \geq 30 \ \wedge salaTemporanea(lp[j-1].sala)) \ \wedge \ I \cap J = \emptyset \ \wedge \\
(\forall h \in I \cup J)^C. \ (lp[h].orario - lp[h-1].orario \leq 30 \lor (!salaPermanente(lp[h-1].sala) \ \wedge \ !salaTemporanea(lp[h-1].sala)))))\}).
\end{gather*}

Ovvero la classe di equivalenza \(\mathbf{X}_{h, k}\) contiene tutti e soli i vettori che corrispondono a visite in cui bisogna pagare per
\textit{k} sale permanenti e \textit{h} sale temporanee indipendentemente dall'ordine in cui queste sono state visitate ponendo dunque che
il valore restituito da \textit{calcolaTariffa} \`e del tipo 3k + 5h.

Si osserva inoltre che senza perdita di generalit\`a si pu\`o supporre di considerare \(lp.length() \leq 3\) poich\'e il comportamento atteso
non dipende dalla lunghezza del vettore in ingresso (oltre una dimensione che permetta almeno un pagamento) ai fini del testing del metodo.

Si fissa dunque \(n = 3\) e si suppone di identificare con \textit{p} un oggetto di tipo Passaggio con i parametri (p.orario, p.sala),
una batteria di test composta da un test di errore e un test di frontiera per ogni classe di equivalenza individuata pu\`o essere definita come:
\begin{itemize}
    \item \{NULL; errore; A\};
    \item k = 0, h = 0 : \{\{[0, 0], [10, 1], [20, 0]\}, 0, A\};
    \item k = 0, h = 1 : \{\{[0, 0], [20, 1], [50, 2]\}, 5, A\};
    \item k = 1, h = 0 : \{\{[0, 0], [30, 1], [50, 0]\}, 3, A\};
    \item k = 1, h = 1 : \{\{[0, 0], [30, 1], [60, 0]\}, 8, A\};
    \item k = 0, h = 2 : \{\{[0, 1], [30, 4], [60, 0]\}, 10, A\};
    \item k = 2, h = 0 : \{\{[0, 0], [30, 3], [60, 0]\}, 6, A\};
\end{itemize}

\subsection{Esercizio 9.b}
Il corpo del metodo viene partizionato in questo modo:
\begin{enumerate}
    \item int tariffa = 0;
    \item for (i = 1;
    \item i \(<\) lp.length;
    \item if (lp[i].orario - lp[i-1].orario \(>=\) 30)
    \item return tariffa;
    \item i++)
    \item if (salaPermanente(lp[i-1].sala))
    \item tariffa += 3;
    \item if (salaTemporanea(lp[i-1].sala))
    \item tariffa += 5;
\end{enumerate}
%\includegraphics{esercizio9.jpg}
\pagebreak

\subsection{Esercizio 9.c}
La batteria di test definita al punto (a) con l'aggiunta dell'ulteriore test \( \{\{[0, 0]\}, 0 , A\} \) \`e sufficiente per garantire una copertura
del 100\% del grafo di flusso precedente. Infatti per garantire tale copertura sono sufficienti i seguenti testcase:
\begin{itemize}
    \item un testcase in cui il corpo del ciclo \textit{for} non viene mai raggiunto (quello appena aggiunto);
    \item un testcase in cui il corpo del ciclo \textit{for} viene eseguito almeno una volta (succede in ogni caso in cui lp \( \neq \) NULL);
    \item un testcase in cui il corpo del primo ramo \textit{if} non viene mai raggiunto (k = h = 0);
    \item un testcase in cui il corpo del primo ramo \textit{if} viene eseguito almeno una volta (k = 1, h = 0);
    \item un testcase in cui il corpo del secondo ramo \textit{if} viene eseguito almeno una volta, il terzo invece no (k = h = 1, iterazione n. 1);
    \item un testcase in cui il corpo del terzo ramo \textit{if} viene eseguito almeno una volta, il secondo invece no (k = h = 1, iterazione n. 2).
\end{itemize}

\subsection{Esercizio 9.d}
La specifica pone il vincolo \( \neg(\exists int \ i \ | \ salaPermanente(i) = salaTemporanea(i) = True) \).

Si propongono due soluzioni all'esercizio affrontando il problema da due punti di vista differenti.


\subsubsection{Prima soluzione}
Suppongo le due implementazioni di \textit{salaPermanente} e di \textit{salaTemporanea} come due funzioni del tipo \( f : R \rightarrow \{T, F\} \).
Ponendo ad esempio ID = \( \infty \ \lor \ \) ID = \( - \infty \) le operazioni usate all'interno delle due funzioni
potrebbero non essere definite in quanto \( \infty \mod{n} \) ha un risultato che dipende dall'implementazione del linguaggio in cui si sviluppa
(e.g. supponendo che i metodi siano implementati in modo simile a quelli dello stub fornito al punto (a), in Java il problema sussiterebbe poich\'e il
modulo di infinito risulta essere \textit{NaN} e - supponendo anche che quella fosse una Sala - entrambi i metodi restituirebbero un valore \textit{False}
che permetterebbe dunque al visitatore di non pagare per quella Sala; la coppia di risposte dello stub sarebbe dunque \{False, False\}).

\subsubsection{Seconda soluzione}
Si suppone che le definizioni di salaPermanente e di salaTemporanea siano mal poste, ad esempio:
\begin{lstlisting}[style=codeStyle]
    bool salaPermanente(int ID)
    {
        return ID % 3 == 0;
    }

    bool salaTemporanea(int ID)
    {
        return ID % 3 == 0;
    }
\end{lstlisting}
in questo modo una sala risulta contemporaneamente permanente e temporanea (si restituisce dunque la coppia \{True, True\}) 
- in contrasto con la specifica. Assumendo che tutti i valori per cui vale \( ID \mod{3} == 0\) siano sale permanenti
e supponendo che un visitatore non attraversi solo queste, nel momento in cui (il visitatore) richieder\`a la possibilit\`a di
procedere al pagamento (n.d.r. con \textbf{tariffa bianca}) l'ammontare effettivo non sarebbe coerente con quanto
il visitatore si aspetterebbe.

\end{document}