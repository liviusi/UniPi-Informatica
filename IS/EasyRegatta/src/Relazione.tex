\documentclass[10pt, openany, landscape]{book}
% Set page margins
\usepackage[top=100pt,bottom=95pt,left=66pt,right=66pt]{geometry}

\usepackage[]{graphicx}
% If multiple images are to be added, a folder (path) with all the images can be added here 
\graphicspath{ {images/} }
 
% All page numbers positioned at the bottom of the page
\usepackage{fancyhdr}
\usepackage{lscape}
\usepackage{adjustbox}
\usepackage{booktabs}
\usepackage{float}
\usepackage{setspace}

\fancyhf{} % clear all header and footers
\fancyfoot[C]{\thepage}
\renewcommand{\headrulewidth}{0pt} % remove the header rule
\pagestyle{fancy}

% Changes the style of chapter headings
\usepackage{titlesec}

\titleformat{\chapter}
   {\normalfont\LARGE\bfseries}{\thechapter.}{1em}{}

% Change distance between chapter header and text
\titlespacing{\chapter}{0pt}{50pt}{2\baselineskip}
\usepackage{titlesec}
\titleformat{\section}
  [hang] % <shape>
  {\normalfont\bfseries\Large} % <format>
  {} % <label>
  {0pt} % <sep>
  {} % <before code>
\renewcommand{\thesection}{} % Remove section references...
\renewcommand{\thesubsection}{\arabic{subsection}} %... from subsections

% Numbered subsections
\setcounter{secnumdepth}{3}

% Prevents LaTeX from filling out a page to the bottom
\raggedbottom

\usepackage{listings}
\usepackage{color}
\usepackage{xcolor}
\usepackage{amsmath}
\singlespace
% Code Listings
\definecolor{vgreen}{RGB}{104,180,104}
\definecolor{vblue}{RGB}{49,49,255}
\definecolor{vorange}{RGB}{255,143,102}
\definecolor{vlightgrey}{RGB}{245,245,245}

\definecolor{codegreen}{rgb}{0,0.6,0}
\definecolor{codegray}{rgb}{0.5,0.5,0.5}
\definecolor{codepurple}{rgb}{0.58,0,0.82}
\definecolor{backcolour}{rgb}{0.95,0.95,0.92}

\lstdefinestyle{codeStyle}{
	language=Java,
    backgroundcolor=\color{backcolour},   
    commentstyle=\color{codegreen},
    keywordstyle=\color{magenta},
    numberstyle=\tiny\color{codegray},
    stringstyle=\color{codepurple},
    basicstyle=\ttfamily\footnotesize,
    breakatwhitespace=false,         
    breaklines=true,                 
    captionpos=b,                    
    keepspaces=true,                 
    numbers=left,                    
    numbersep=5pt,                  
    showspaces=false,                
    showstringspaces=false,
    showtabs=false,                  
    tabsize=2
}

\begin{document}

\begin{titlepage}
	\clearpage\thispagestyle{empty}
	\centering
	\vspace{1cm}

    \includegraphics[scale=0.60]{unipi-logo.png}
    
	{\normalsize \noindent Dipartimento di Informatica \\
	             Corso di Laurea in Informatica \par}
	
	\vspace{2cm}
	{\Huge \textbf{EasyRegatta} \par}
	\vspace{1cm}
	{\large Ingegneria del Software a.a. 2020/2021}

    \begin{minipage}[t]{0.47\textwidth}
    	{\large{ Prof.ssa Roberta Gori}}
    \end{minipage}\hfill\begin{minipage}[t]{0.47\textwidth}\raggedleft
        {\large {F D L - ------ \\ }}
        {\large {A F - ------ \\ }}
    	{\large {Giacomo Trapani - ------ \\ }}
    \end{minipage}

    \vspace{4cm}

	\pagebreak

\end{titlepage}

% A * after the section/chapter command indicates an unnumbered header which will not be added to the table of contents

\section{Esercizio 1}
Il diagramma dei casi d'uso descrive le funzionalit\`a che il sistema mette a disposizione all'esterno.
Degni di nota sono il caso d'uso relativo alle comunicazioni - ("\textbf{Effettua una comunicazione}") - che prevede che il sistema
faccia da tramite per le comunicazioni tra imbarcazioni, alle informazioni sui tempi di gara - (i.e. "\textbf{Registrazione
tempo di arrivo}", "\textbf{Comunica passaggio in boa}") - che implicano che il sistema venga utilizzato per segnare in tempo reale
i dati riguardo le posizioni in classifica, alla gestione di una singola gara - ("\textbf{Gestione gara}") - che astrae le
features necessarie a descrivere, condurre e terminare una gara.
\pagebreak
\subsubsection*{Narrativa.}
Si fornisce la narrativa del caso d'uso "\textbf{Modifica bando}".
\begin{table}[H]
    \begin{tabular}{@{}|l|l|@{}}
    \toprule
    \textbf{Nome}                              & Modifica bando.                                                                                                                                                                      \\ \midrule
    \textbf{Breve Descrizione}                 & Effettua la modifica del bando aggiornando il catalogo, invia una notifica a tutte le \\
    ~       & barche che si erano iscritte prima della modifica.                                          \\ \midrule
    \textbf{Attori Primari}                    & Autorità Organizzatrice.                                                                                                                                                             \\ \midrule
    \textbf{Attori Secondari}                  & Nessuno.                                                                                                                                                                             \\ \midrule
    \textbf{Precondizioni}                     & Il bando esiste all’interno del Sistema.                                                                                                                                             \\ \midrule
    \textbf{Sequenza degli eventi principale}  & \begin{tabular}[c]{@{}l@{}}Accerta che i nuovi dati siano corretti.\\ Modifica il bando.\\ Aggiorna il catalogo.\\ Notifica la modifica del bando a tutti i suoi iscritti.\end{tabular} \\ \midrule
    \textbf{Postcondizioni}                    & Il bando è stato modificato, tutte le barche che si sono iscritte prima della modifica \\
    ~       &  hanno ricevuto una notifica.                                                                   \\ \midrule
    \textbf{Sequenze alternative degli eventi} & Dati non corretti, modifica annullata.                                                                                                                                              \\ \bottomrule
    \end{tabular}
    \end{table}
\pagebreak
\section{Esercizio 2}
Viene fornito il diagramma di sequenza per "\textbf{Effettua una comunicazione}". Si sceglie di ripetere il tentativo di
avviare una comunicazione con l'imbarcazione destinataria fino a che il mittente non riesce a connettersi rendendo il
meccanismo quanto pi\`u possibile trasparente all'utente.

\pagebreak

\section{Esercizio 3}
All'interno del diagramma delle classi vengono descritte le entit\`a fondanti il dominio del progetto. Si sceglie - coerentemente
con quanto descritto all'interno del diagramma dell'esercizio precedente - di modellare le comunicazioni come un qualcosa che
riguarda solo le imbarcazioni coinvolte.

\pagebreak

\section{Esercizio 4}
All'interno del diagramma si riconosce come stato fondamentale (in quanto quello di interesse ai fini dell'esercizio)
"\textbf{Gara}", vero e proprio nodo nevralgico per tutte le interazioni che una imbarcazione pu\`o avere (i.e. effettuare
comunicazioni, inviare proteste, inviare il tempo di passaggio alle boe e tagliare il traguardo). Seguendo la specifica, si
sceglie di terminare la vita di una imbarcazione (relativamente a una regata) solo dopo un'ora dal momento in cui si \`e
tagliato il traguardo poich\'e si deve permettere l'invio di proteste anche dopo l'arrivo (ma comunque entro un'ora da questo):
per questo motivo si sceglie di adoperare uno stato "\textbf{Idle}" in modo da non uscire da "\textbf{Arrivato}"
fino al timeout. La stessa logica si utilizza anche per segnare una imbarcazione come "\textit{Non partita}" - flag che necessita
di un altro timeout - a patto che siano passati 30 minuti dal momento in cui \`e arrivato il segnale di "\textit{Partenza}"
e questa non sia - appunto - ancora partita.

\pagebreak

\section{Esercizio 5}
Si modellano le attivit\`a svolte dall'indizione del bando alla fine della regata.
La prima sottoattivit\`a svolta - "\textbf{Bando}" - riguarda l'indizione di un bando e le operazioni che ne vengono implicate
(i.e. la gestione delle iscrizioni, la possibilit\`a di modificare un bando a patto che non si sia ancora raggiunto il termine
delle iscrizioni); una volta nominato il comitato di gara (evento che si suppone avvenga al termine delle iscrizioni al bando
e prima della regata) si passa a una seconda sottoattivit\`a - "\textbf{Gara}" - che modella lo svolgimento della regata (dalla
gestione delle proteste a quella delle comunicazioni tra imbarcazioni, dai segnali di partenza a quelli per il passaggio alle boe);
si conclude stilando la classifica delle imbarcazioni un'ora dopo il tempo limite.

\pagebreak

\section{Esercizio 6}
\subsubsection*{Vista componenti-connettori.}
Si sceglie di dividere il sistema software \textit{EasyRegatta} in due componenti fondamentali.
La prima - "\textbf{GestioneBandi}" - si occupa della pubblicazione dei bandi (e il loro salvataggio) e permette agli
utenti di consultarne comunicando con il database a questi dedicato - "\textbf{DataBaseBandi}" - abilitandone, inoltre, l'iscrizione
attraverso "\textbf{AppIscrizioni}", componente con la quale si interfaccia anche per inviare notifiche agli (utenti) iscritti
in seguito a modifiche; la seconda - "\textbf{GestioneGare}" - si occupa di seguire per intero una regata (i.e. stilando
la classifica, permettendo di esaminare le proteste etc.), comunica con un database - "\textbf{DataBaseGare}" - nel quale
salva i risultati della competizione e con "\textbf{AppComunicazioni}", utilizzata per gestire le proteste (la stessa componente
viene utilizzata anche per le comunicazioni tra imbarcazioni senza che queste passino dalla componente centrale).

\pagebreak

\subsubsection*{Vista di dislocazione.}
A livello hardware si identificano 4 differenti ambienti di esecuzione: il server centrale - "\textbf{ApplicationServer}" -
utilizzato anche per le basi di dati mantenute dal sistema; il dispositivo di un utente - "\textbf{DispositivoUtente}" -
(e.g. uno smartphone, un computer etc.) su cui viene eseguito l'applicativo che consente l'iscrizione a un bando; il dispositivo
utilizzato dal comitato di regata - "\textbf{ComputerComitato}" - che viene utilizzato durante lo svolgimento della gara (e.g. pu\`o
essere utilizzato per la gestione delle proteste); il dispositivo utilizzato all'interno delle imbarcazioni -
"\textbf{ComputerBarca}" - utilizzato per la gestione delle comunicazioni (tra imbarcazioni).

\pagebreak

\section{Esercizio 7}
Per impostare il test, si individuano delle categorie secondo le quali partizionare i valori in input:
\begin{itemize}
\itemsep0em
    \item nome := [a-zA-Z][a-zA-Z ]+ \textbar \ "" \textbar \ null \textbar \ NOT([a-zA-Z][a-zA-Z ]+). Il primo pattern corrisponde alle stringhe che contengono solo spazi, lettere maiuscole e lettere minuscole (e che non iniziano con uno spazio). Si considera valido solo se (il match col primo pattern) ha successo. La notazione utilizzata per l'ultima \`e per identificare tutte le stringhe che falliscono il match con la prima espressione regolare.
    \item rating := \(\{x : x \in Z \ \wedge \ x \in [-\infty; 0]\} \) \textbar \ \(\{x : x \in N \ \wedge \ x \neq 0\} \). Si considera valido solo per valori maggiori di zero.
    \item temporeale = \(\{x : x \in Z \ \wedge \ x \in [-\infty; 0]\} \) \textbar \ \(\{x : x \in N \ \wedge \ x \neq 0\} \). Si considera valido solo per valori maggiori di zero.
    \item distanza = \(\{x : x \in Z \ \wedge \ x \in [-\infty; 0]\} \) \textbar \ \(\{x : x \in N \ \wedge \ x \neq 0\} \). Si considera valido solo per valori maggiori di zero.
    \item Tempo Compensato = \(\{x : x \in Z \ \wedge \ x \in [-\infty; 0]\} \) \textbar \ \(\{x : x \in N \ \wedge \ x \neq 0\} \). Si considera valido solo per valori maggiori di zero.
    \item rep[i] = null \textbar \ definito. Si considera valido se tutti i campi sono validi.
    \item rep = null \textbar \  [] \textbar \ array con almeno un elemento. Si considera valido se tutti gli elementi sono validi.
\end{itemize}
Si distinguono dunque 384 possibili combinazioni, tra queste se ne scelgono 11, una per ogni caso non valido che non viene
gestito + una valida. Si sceglie un criterio \textit{white box}.

Si scrive la seguente batteria di test:

\begin{table}[H]
    \scalebox{0.85}{
    \begin{tabular}{@{}|l|l|@{}}
    \toprule
    \textbf{Input}                                                                & \textbf{output atteso}                 \\ \midrule
    \{\textless{}null, 5\textgreater{}\}                                          & errore: array null.                    \\ \midrule
    \{\textless{}{[}{]},3\textgreater{}\}                                         & errore: array vuoto.                   \\ \midrule
    \{\textless{}{[}\{“Luca”, 4.5, 10\}, \{“Antonio, 5.5, 11\}, \{“Mario”, 1.2, 23\}{]}, -8\textgreater{}\}        & errore: distanza non positiva.              \\ \midrule
    \{\textless{}{[}\{“Fabio”, 2.3, 20\}, \{null, 4.1, 10\}{]}, 4\textgreater{}\} & errore: rep{[}1{]}.nome null.          \\ \midrule
    \{\textless{}{[}\{“Francesco”,-1.3,23\}{]}, \{“Alberto”,3.2, 14\}{]}, 2\textgreater{}\}                        & errore: rep{[}0{]}.rating non positivo.     \\ \midrule
    \{\textless{}{[}\{“Luca”, 4.5, -3\}, \{“Antonio, 5.5, 11\}, \{“Mario”, 1.2, 23\}{]}, 4\textgreater{}\}         & errore: rep{[}0{]}.tempoReale non positivo. \\ \midrule
    \{\textless{}{[}\{“Aldo”, 4.5, 30\}, \{“”, 3.4, 25\}{]}, 4\textgreater{}\}    & errore: rep{[}1{]}.nome vuoto.         \\ \midrule
    \{\textless{}{[}\{“Marco”, 2.5, 31\}, \{“!!?”, 3.5, 21\}{]}, 2\textgreater{}\}                                 & errore: rep{[}1{]}.nome non valido.         \\ \midrule
    \{\textless{}{[}\{“Giovanni”, 9.1, 35\}, null{]}, 6\textgreater{}\}           & errore: rep{[}1{]} null.               \\ \midrule
    \{\textless{}{[}\{”Federico”, 2.5, 52\}{]}, 10\textgreater{}\}                & errore: Tempo Compensato non positivo. \\ \midrule
    \{\textless{}{[}\{“Giacomo”, 4.5, 40\}, \{“Alberto”, 1.2, 20\}, \{“Francesco”, 2.3, 45\}{]}, 8\textgreater{}\} & “Giacomo”.                                  \\ \bottomrule
    \end{tabular}
    }
    \end{table}
Si implementa il seguente test driver:

\begin{lstlisting}[style=codeStyle]
bool testDriver()
{
    Report [] r = new Report[3];
    r[0] = new Report("Giacomo", 4.5, 40);
    r[1] = new Report("Alberto", 1.2, 20);
    r[2] = new Report("Francesco", 2.3, 45);
    String v = calcolaVincitore(r, 8);
    return v.equals("Giacomo");
}
\end{lstlisting}

\end{document}