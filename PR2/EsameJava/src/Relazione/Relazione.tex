\documentclass[10pt, italian, openany]{book}

% Set page margins
\usepackage[top=100pt,bottom=100pt,left=68pt,right=66pt]{geometry}

\usepackage[]{graphicx}
% If multiple images are to be added, a folder (path) with all the images can be added here 
\graphicspath{ {images/} }

\usepackage{hyperref}
\hypersetup{
    colorlinks=true,
    linkcolor=blue,
    filecolor=magenta,      
    urlcolor=blue,
}
 
% All page numbers positioned at the bottom of the page
\usepackage{fancyhdr}
\fancyhf{} % clear all header and footers
\fancyfoot[C]{\thepage}
\renewcommand{\headrulewidth}{0pt} % remove the header rule
\pagestyle{fancy}

% Changes the style of chapter headings
\usepackage{titlesec}

\titleformat{\chapter}
   {\normalfont\LARGE\bfseries}{\thechapter.}{1em}{}

% Change distance between chapter header and text
\titlespacing{\chapter}{0pt}{50pt}{2\baselineskip}
\usepackage{titlesec}
\titleformat{\section}
  [hang] % <shape>
  {\normalfont\bfseries\Large} % <format>
  {} % <label>
  {0pt} % <sep>
  {} % <before code>
\renewcommand{\thesection}{} % Remove section references...
\renewcommand{\thesubsection}{\arabic{subsection}} %... from subsections

% Numbered subsections
\setcounter{secnumdepth}{3}

% Prevents LaTeX from filling out a page to the bottom
\raggedbottom

\usepackage{listings}
\usepackage{color}
\usepackage{xcolor}
\usepackage{amsmath}
% Code Listings
\definecolor{vgreen}{RGB}{104,180,104}
\definecolor{vblue}{RGB}{49,49,255}
\definecolor{vorange}{RGB}{255,143,102}
\definecolor{vlightgrey}{RGB}{245,245,245}
\lstdefinestyle{bash} {
	language=bash,
	basicstyle=\ttfamily,
	keywordstyle=\color{vblue},
	identifierstyle=\color{black},
	commentstyle=\color{vgreen},
	tabsize=4,
	backgroundcolor = \color{vlightgrey},
	literate=*{:}{:}1
}

\definecolor{codegreen}{rgb}{0,0.6,0}
\definecolor{codegray}{rgb}{0.5,0.5,0.5}
\definecolor{codepurple}{rgb}{0.58,0,0.82}
\definecolor{backcolour}{rgb}{0.95,0.95,0.92}

\lstdefinestyle{codeStyle}{
	language=Java,
    backgroundcolor=\color{backcolour},   
    commentstyle=\color{codegreen},
    keywordstyle=\color{magenta},
    numberstyle=\tiny\color{codegray},
    stringstyle=\color{codepurple},
    basicstyle=\ttfamily\footnotesize,
    breakatwhitespace=false,         
    breaklines=true,                 
    captionpos=b,                    
    keepspaces=true,                 
    numbers=left,                    
    numbersep=5pt,                  
    showspaces=false,                
    showstringspaces=false,
    showtabs=false,                  
    tabsize=2
}

\begin{document}

\begin{titlepage}
	\clearpage\thispagestyle{empty}
	\centering
	\vspace{1cm}

    \includegraphics[scale=0.60]{unipi-logo.png}
    
	{\normalsize \noindent Dipartimento di Informatica \\
	             Corso di Laurea in Informatica \par}
	
	\vspace{2cm}
	{\Huge \textbf{Relazione progetto Java} \par}
	\vspace{1cm}
	{\large Programmazione II}
	\vspace{5cm}

    \begin{minipage}[t]{0.47\textwidth}
    	{\large{ Prof. Gianluigi Ferrari}}
    \end{minipage}\hfill\begin{minipage}[t]{0.47\textwidth}\raggedleft
    	{\large {Giacomo Trapani \\ 600124 - Corso A\\ }}
    \end{minipage}

    \vspace{4cm}

	{\normalsize Anno Accademico 2020/2021 \par}

	\pagebreak

\end{titlepage}

\chapter*{Relazione}

% A * after the section/chapter command indicates an unnumbered header which will not be added to the table of contents
\section{Dettagli sul metodo di sviluppo}
Il progetto \`e stato sviluppato con il supporto del text editor \textit{VSCode}, senza tuttavia l'utilizzo di strumenti esterni quali \textit{Maven} per la compilazione o \textit{JUnit} per la creazione dei tests, che invece sono stati scritti a mano. Di conseguenza, l'esecuzione \`e possibile utilizzando semplicemente i comandi messi a disposizione dal JDK. Ad esempio:
\begin{lstlisting}[style=bash]
cd src
javac *.java
java -ea Main
\end{lstlisting}
In alternativa si pu\`o utilizzare il bash script in allegato:
\begin{lstlisting}[style=bash]
cd src
chmod +x run.sh
./run.sh
\end{lstlisting}

\section{Dettagli implementativi}
Il progetto consiste nello sviluppo di una componente software di supporto alla gestione e l'analisi di un social network denominato \textbf{MicroBlog}. Nella rete sociale \`e consentito inviare messaggi di testo composti da al pi\`u 140 caratteri, chiamati \textbf{Post}. Agli utenti \`e consentito seguire i Post (implementato con un meccanismo di Like) e segnalarli. 

\subsection{Parte 1}
Il tipo di dato \textbf{MyPost} rappresenta una collezione immutabile di dati e implementa l'interfaccia \textbf{Post}. Un MyPost \`e definito dai seguenti elementi:
\begin{lstlisting}[style=codeStyle]
private final long ID; // identificatore univoco del Post
private final String Author; // autore del Post
private final String Text; // contenuto del Post
private final String Timestamp; // timestamp del Post
\end{lstlisting}
I parametri vengono implementati come \textit{final} per vietarne la modifica; sono privati ed accessibili attraverso dei getter specifici dichiarati nell'interfaccia \textbf{Post}.
Sui parametri valgono le condizioni indicate nell'invariante di rappresentazione:
\begin{gather*}
    ID \geq 0 \ \wedge \ Author \neq null \ \wedge \ Text \neq null \ \wedge \ Timestamp \neq null \\
    \wedge \ \neg (Author \in \ "/\backslash A\backslash s*\backslash z/") \ \wedge \ \neg (Author.isEmpty())  \ \wedge \ \neg (Text.isEmpty()).
\end{gather*}
A garantire l'unicit\`a dell'identificatore univoco \textit{ID} non \`e la classe MyPost - che rappresenta solamente un contenitore di dati - bens\'i la classe che lavora sui Post: \textbf{MySocialNetwork}.

\subsection{Parte 2}
Il tipo di dato \textbf{MySocialNetwork} rappresenta una collezione mutabile di dati e implementa l'interfaccia \textbf{SocialNetwork}. La progettazione richiede per\`o l'utilizzo di un tipo di dato \textbf{Like}:
in questa implementazione questo viene considerato un \textbf{Post} istanziato con un testo del tipo "\(Like: \langle id \rangle\)" con id definito come l'identificativo del Post a cui l'autore aggiunge un Like.
Da questo punto in poi l'insieme dei Post viene considerato come formato dall'unione degli insiemi \textbf{textualPosts} (ossia i Post che non sono Like) e \textbf{likePosts} (ossia i Post che sono Like).
Su ogni elemento di likePosts vale la seguente condizione:
\[ (\forall Post \ p \in likePosts. \ (\exists Post \ q \in textualPosts \ | \ q.getID() = ReferencedID(p.getText()))). \]
La funzione \(ReferencedID(String \ s)\) \`e banalmente definita come la funzione che estrae la stringa che corrisponde all'identificativo a cui il testo del Post fa riferimento.

Per fare in modo che un Post l sia considerabile un Like valido deve inoltre verificarsi la seguente condizione:
\begin{gather*}
(\exists Post \ p \in textualPosts \ | \ getReferencedID(l.getText()) = p.getID() \ \wedge \ p.getAuthor() \neq l.getAuthor()) \\
\ \wedge \ \neg(\exists Post \ q \in likePosts \ | \ getReferencedID(q.getText()) = getReferencedID(l.getText()) \\
\ \wedge \ q.getAuthor() = l.getAuthor())
\end{gather*}

Nonostante \(Post \ p \in textualPosts\) e \(Post \ q \in likePosts\) siano implementati dallo stesso tipo di dato, i metodi che lavorano sul contenuto dei Post ignorano \textit{q} che non si ritiene rilevante.

\subsubsection*{MySocialNetwork}
Il tipo di dato MySocialNetwork \`e definito dalle seguenti variabili di istanza:
\begin{lstlisting}[style=codeStyle]
// contatore per i Post istanziati all'interno del social network
private long postsNum; 
// Follows[a] = insieme degli utenti seguiti da a
protected Map<String, Set<String>> Follows;
// insieme textualPosts
private Set<Post> textualPosts;
// insieme likePosts
private Set<Post> likePosts;
// Followers[a] = insieme degli utenti che seguono a
protected Map<String, Set<String>> Followers;
\end{lstlisting}
Sulla classe vale il seguente invariante di rappresentazione:
\begin{gather*}
Follows \neq null \ \wedge \ textualPosts \neq null \ \wedge \ likePosts \neq null  \ \wedge \ Followers \neq null \\
\ \wedge \ postsNum =  textualPosts.size() + likePosts.size() \\
\ \wedge \ ( \forall String \ i \in [0; Follows.size()]. \ i \neq null \ \wedge \ Follows[i] \neq null ) \\
\ \wedge \ ( \forall Integer \ i \in [0; Followers.size()]. \ i \neq null \ \wedge \  Followers[i] \neq null) \\
\ \wedge \ (\forall Post \ p \in textualPosts. \ p \neq null) \\
\ \wedge \ (\forall Post \ p \in likePosts. \ p \neq null) \\
\ \wedge \ ( \forall Pair \ \langle k, v \rangle \in Follows.Entry. \ (\forall String \ u \in v. \ (\exists Post \ p \ | \ p \in textualPosts \ \wedge \ p.getAuthor() = u \\
\ \wedge \ ( \exists Post \ q \ | \ q \in likePosts \ \wedge \ q.getAuthor = k \ \wedge \ getReferencedID(q.getText()) = p.getID())))) \\
\ \wedge \ ( \forall Pair \ \langle k, v \rangle \in Followers.Entry. \ (\forall String \ u \in v. \ (\exists Post \ p \ | \ p \in textualPosts \ \wedge \ p.getAuthor() = k \\
\ \wedge \ ( \exists Post \ q \ | \ q \in likePosts \ \wedge \ q.getAuthor = u \ \wedge \ getReferencedID(q.getText()) = p.getID())))). \\
\end{gather*}
L'invariante assicura - oltre all'assenza di parametri di valore \textit{null} - che siano verificate tutte le condizioni necessarie nel meccanismo 
di follows implementato: un Utente (identificato in modo univoco da un parametro di tipo \textbf{String}) u segue un Utente v se e solo se 
esiste (almeno) un Like di u a un Post di v; viene mantenuto (per induzione) da due chiamate alla funzione \textit{maintainIR} (una per la mappa dei \textit{Follows}, l'altra per i \textit{Followers}) ogni volta che \textit{this} viene modificato:
\begin{lstlisting}[style=codeStyle]
private static void maintainIR(Map<String, Set<String>> map, String author, String s)
{
    Set<String> tmp = new HashSet<>();
    if (map.get(s) != null)
        tmp = map.get(s);
    if (!tmp.contains(author))
    {
        tmp.add(author);
        map.put(s, tmp);
    }
}
\end{lstlisting}
La funzione viene chiamata ogni volta che viene creato un Like, aggiunge \textit{author} all'insieme dei valori mappati dalla chiave\textit{s}.

Viene scelto di dividere in due sottoinsiemi i Post presenti in MySocialNetwork per ridurre la complessit\`a del metodo addPost:
\begin{lstlisting}[style=codeStyle]
public Post addPost(String author, String text)
    throws NullPointerException, Post.IllegalPostException
{
    Post p = null;
    try
    {
        p = new MyPost(postsNum, author, text);
    }
    catch (NullPointerException | Post.IllegalPostException e)
    {
        throw e;
    }
    if (SocialNetwork.LikePattern(text))
    {
        try
        {
            return createLike(p);
        }
        catch (Post.IllegalPostException e)
        {
            throw e;
        }
    }
    if (textualPosts.add(p))
    {
        postsNum++;
        SocialNetwork.initializeUser(Follows, Followers, p.getAuthor());
    }
    return p;
}
\end{lstlisting}
Nel caso in cui il Post che si vuole aggiungere all'insieme dei Post presenti nella rete non segue il formato di un Like questo metodo ha complessit\`a \( O(LikePattern(text))\) 
(ossia un tempo costante); nel caso contrario, la complessit\`a del metodo (al caso pessimo) diventa \(O(textualPosts.size()) + O(likePosts.size())\) 
poich\'e vengono vietate le operazioni di Like su un proprio Post, uno non presente nel Social, uno a cui lo si \`e gi\`a messo, o su un altro Like - reputate illogiche.
In entrambi i casi alla complessit\`a del metodo va aggiunto anche il tempo necessario per mantenere l'invariante della classe.

Si sceglie di mantenere la variabile di istanza \textit{Followers} - corrispondente all'inversa della Map \textit{Follows} - per ridurre al minimo la complessit\`a del metodo influencers:
\begin{lstlisting}[style=codeStyle]
public List<String> influencers()
{
    List<Entry<String, Set<String>>> list = new LinkedList<>(Followers.entrySet());
    Collections.sort(list, new Comparator<Entry<String, Set<String>>>()
    {
        public int compare(Entry<String, Set<String>> e1, Entry<String, Set<String>> e2)
        { return Integer.compare(e1.getValue().size(), e2.getValue().size()); }
    });
    List<String> res = new LinkedList<>();
    for (Entry<String, Set<String>> l: list) res.add(l.getKey());
    Collections.reverse(res); // now it is in a descending order
    return res;
}
\end{lstlisting}
Con questa implementazione il metodo \textit{influencers} non deve andare a operare sull'insieme dei Post o sulla Map dei Follows: diventa un metodo di sorting con complessit\`a \(O(n \times \log n), \ con \ n := Followers.entrySet().size()\).

Nel Social Network \`e presente anche un meccanismo di \textbf{tagging}: \`e possibile menzionare altri utenti presenti nel Social Network all'interno di un generico \(Post \ p \ | \ p \in textualPosts\) con testo contenente una sottostringa del tipo \(@Username \ | \ Username \in Users \ \wedge \ p.getAuthor() \neq Username, \ con \ Users := Follows.keySet()\) (con la funzione \(getTagged()\) che banalmente restituisce il nome utente dell'utente menzionato); non viene proibita la creazione di un \(Post \ q \ | \ q.getAuthor() = q.getTagged()\), in casi di questo tipo nel testo non sar\`a rilevata alcuna menzione. 
Questo meccanismo \`e implementato dai metodi che vanno a restituire l'insieme degli utenti menzionati all'interno dei Post; se ne riporta un valido esempio:
\begin{lstlisting}[style=codeStyle]
private Set<String> detectMention(Post p)
    throws NullPointerException
{
    if (p == null)
        throw new NullPointerException("Post cannot be null.\n");
    Set<String> res = new HashSet<>();
    String delims = "[ ]";
    String[] tokens = p.getText().split(delims);
    for (String t : tokens)
        if (t.startsWith("@")) 
            if (!(p.getAuthor().equals(t.substring(1))) && Follows.containsKey(t.substring(1)))
                res.add(t.substring(1));
    return res;
}

public Set<String> getMentionedUsers()
{
    Set<String> res = new HashSet<>();
    for (Post p : textualPosts)
        res.addAll(detectMention(p));
    return res;
}
\end{lstlisting}

\subsection{Parte 3}
Il tipo di dato \textbf{MyKiddiesSocialNetwork} - collezione mutabile di dati - rappresenta una estensione gerarchica del tipo di dato \textbf{MySocialNetwork} in cui viene implementato un meccanismo di \textbf{Report} (o \textbf{Segnalazione}) e implementa l'interfaccia \textbf{KiddiesSocialNetwork}. La progettazione richiede per\`o l'utilizzo di un tipo di dato \textbf{Report}:
in questa implementazione questo viene considerato un \textbf{Post} istanziato con un testo del tipo: "\(Segnalazione: \langle id \rangle\)" con id definito come l'identificativo del Post che l'autore vuole segnalare.
Da questo punto in poi l'insieme dei Post viene considerato come formato dall'unione degli insiemi \textbf{textualPosts} (ossia i Post che non sono n\'e Like n\'e Report), \textbf{likePosts} (ossia i Post che sono Like) e \textbf{reportPosts}.
Su ogni elemento di reportPosts vale la seguente condizione:
\[ (\forall Post \ p \in reportPosts. \ (\exists Post \ q \in textualPosts \ | \ q.getID() = ReferencedID(p.getText()))). \]
Per fare in modo che un Post r sia considerabile un Report valido deve inoltre verificarsi la seguente condizione:
\begin{gather*}
(\exists Post \ p \in textualPosts \ | \ getReferencedID(l.getText()) = p.getID() \ \wedge \ \neg (p \in "^\wedge [A-Za-z]*") \\ 
\ \wedge \ p.getAuthor() \neq l.getAuthor()) \\
\ \wedge \ \neg(\exists Post \ q \in reportPosts \ | \ getReferencedID(q.getText()) = getReferencedID(l.getText()) \\
\ \wedge \ q.getAuthor() = l.getAuthor()).
\end{gather*}

\subsubsection{MyKiddiesSocialNetwork}
Il tipo di dato MyKiddiesSocialNetwork \`e definito dalla seguente variabile di istanza (a cui vanno aggiunte quelle di MySocialNetwork):
\begin{lstlisting}[style=codeStyle]
private Set<Post> reportPosts; // insieme reportPosts
\end{lstlisting}
Sulla classe vale il seguente invariante di rappresentazione:
\[ super.IR() \ \wedge \ reportPosts \neq null \ \wedge \ (\forall Post \ p \in reportPosts. \ p \neq null). \]

Poich\'e la specifica del progetto non prevede un meccanismo di gestione dei post che hanno ricevuto segnalazioni si sceglie di implementare sia un getter per i post segnalati sia per quelli "puliti":
\begin{lstlisting}[style=codeStyle]
public Set<Post> getReportedPosts()
{
    Set<Post> src = new HashSet<>(getPosts());
    Set<Post> res = new HashSet<>();
    for (Post p : getSegnalazioni())
    {
        for (Post q : src)
            if (SocialNetwork.getReferencedID(p.getText()).equals(q.getID()))
            {
                src.remove(q);
                res.add(q);
                break;
            }
    }
    return res;
}

public Set<Post> getFilteredPosts()
{
    Set<Post> res = new HashSet<>(getPosts());
    res.removeAll(getReportedPosts());
    return res;
}
\end{lstlisting}


\section{Test}
Il \textbf{file di test} \`e stato scritto a mano, senza l'utilizzo di framework esterni. Utilizza una stessa funzione per operare sulle due classi, 
in modo da assicurare che il principio di sostituzione venga rispettato. Per ogni funzione, sono stati controllati anche i relativi \textbf{edge cases}, 
utilizzando scenari limite per lanciare le varie eccezioni. L'esito di ogni test viene stampato nel file "out.txt" su cui viene reindirizzato l'output:
in questo si possono vedere le linee in cui le eccezioni sono state correttamente lanciate e - alla fine del file - si troveranno dei commenti accompagnati da una spunta verde ("✅")
nel caso in cui i test siano stati passati con successo.

\end{document}